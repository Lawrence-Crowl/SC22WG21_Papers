\documentclass[ebook,11pt,article]{memoir}
\usepackage{geometry}  % See geometry.pdf to learn the layout options. There are lots.
\geometry{a4paper}  % ... or a4paper or a5paper or ... 
%\geometry{landscape}  % Activate for for rotated page geometry
%\usepackage[parfill]{parskip}  % Activate to begin paragraphs with an empty line rather than an indent


\usepackage[final]
           {listings}     % code listings
\usepackage{color}        % define colors for strikeouts and underlines
\usepackage{underscore}   % remove special status of '_' in ordinary text
\usepackage{xspace}
\usepackage[normalem]{ulem} % for insertion marks with \uline
\usepackage{url}

\pagestyle{myheadings}

\newcommand{\papernumber}{p0448r0}
\newcommand{\paperdate}{2016-10-14}

\markboth{\papernumber{} \paperdate{}}{\papernumber{} \paperdate{}}

\title{\papernumber{} - A strstream replacement using span\textless{}charT\textgreater{} as buffer}
\author{Peter Sommerlad}
\date{\paperdate}                % Activate to display a given date or no date
% Definitions and redefinitions of special commands

%%--------------------------------------------------
%% Difference markups
\definecolor{addclr}{rgb}{0,.6,.3} %% 0,.6,.6 was to blue for my taste :-)
\definecolor{remclr}{rgb}{1,0,0}
\definecolor{noteclr}{rgb}{0,0,1}

\renewcommand{\added}[1]{\textcolor{addclr}{\uline{#1}}}
\newcommand{\removed}[1]{\textcolor{remclr}{\sout{#1}}}
\renewcommand{\changed}[2]{\removed{#1}\added{#2}}

\newcommand{\nbc}[1]{[#1]\ }
\newcommand{\addednb}[2]{\added{\nbc{#1}#2}}
\newcommand{\removednb}[2]{\removed{\nbc{#1}#2}}
\newcommand{\changednb}[3]{\removednb{#1}{#2}\added{#3}}
\newcommand{\remitem}[1]{\item\removed{#1}}

\newcommand{\ednote}[1]{\textcolor{noteclr}{[Editor's note: #1] }}
% \newcommand{\ednote}[1]{}

\newenvironment{addedblock}
{
\color{addclr}
}
{
\color{black}
}
\newenvironment{removedblock}
{
\color{remclr}
}
{
\color{black}
}

%%--------------------------------------------------
%% Sectioning macros.  
% Each section has a depth, an automatically generated section 
% number, a name, and a short tag.  The depth is an integer in 
% the range [0,5].  (If it proves necessary, it wouldn't take much
% programming to raise the limit from 5 to something larger.)


% The basic sectioning command.  Example:
%    \Sec1[intro.scope]{Scope}
% defines a first-level section whose name is "Scope" and whose short
% tag is intro.scope.  The square brackets are mandatory.
\def\Sec#1[#2]#3{%
\ifcase#1\let\s=\chapter
      \or\let\s=\section
      \or\let\s=\subsection
      \or\let\s=\subsubsection
      \or\let\s=\paragraph
      \or\let\s=\subparagraph
      \fi%
\s[#3]{#3\hfill[#2]}\label{#2}}

% A convenience feature (mostly for the convenience of the Project
% Editor, to make it easy to move around large blocks of text):
% the \rSec macro is just like the \Sec macro, except that depths 
% relative to a global variable, SectionDepthBase.  So, for example,
% if SectionDepthBase is 1,
%   \rSec1[temp.arg.type]{Template type arguments}
% is equivalent to
%   \Sec2[temp.arg.type]{Template type arguments}
\newcounter{SectionDepthBase}
\newcounter{scratch}

\def\rSec#1[#2]#3{%
\setcounter{scratch}{#1}
\addtocounter{scratch}{\value{SectionDepthBase}}
\Sec{\arabic{scratch}}[#2]{#3}}

\newcommand{\synopsis}[1]{\textbf{#1}}

%%--------------------------------------------------
% Indexing

% locations
\newcommand{\indextext}[1]{\index[generalindex]{#1}}
\newcommand{\indexlibrary}[1]{\index[libraryindex]{#1}}
\newcommand{\indexgram}[1]{\index[grammarindex]{#1}}
\newcommand{\indeximpldef}[1]{\index[impldefindex]{#1}}

\newcommand{\indexdefn}[1]{\indextext{#1}}
\newcommand{\indexgrammar}[1]{\indextext{#1}\indexgram{#1}}
\newcommand{\impldef}[1]{\indeximpldef{#1}implementation-defined}

% appearance
\newcommand{\idxcode}[1]{#1@\tcode{#1}}
\newcommand{\idxhdr}[1]{#1@\tcode{<#1>}}
\newcommand{\idxgram}[1]{#1@\textit{#1}}

%%--------------------------------------------------
% General code style
\newcommand{\CodeStyle}{\ttfamily}
\newcommand{\CodeStylex}[1]{\texttt{#1}}

% Code and definitions embedded in text.
\newcommand{\tcode}[1]{\CodeStylex{#1}}
\newcommand{\techterm}[1]{\textit{#1}\xspace}
\newcommand{\defnx}[2]{\indexdefn{#2}\textit{#1}\xspace}
\newcommand{\defn}[1]{\defnx{#1}{#1}}
\newcommand{\term}[1]{\textit{#1}\xspace}
\newcommand{\grammarterm}[1]{\textit{#1}\xspace}
\newcommand{\placeholder}[1]{\textit{#1}}
\newcommand{\placeholdernc}[1]{\textit{#1\nocorr}}

%%--------------------------------------------------
%% allow line break if needed for justification
\newcommand{\brk}{\discretionary{}{}{}}
%  especially for scope qualifier
\newcommand{\colcol}{\brk::\brk}

%%--------------------------------------------------
%% Macros for funky text
\newcommand{\Cpp}{\texorpdfstring{C\kern-0.05em\protect\raisebox{.35ex}{\textsmaller[2]{+\kern-0.05em+}}}{C++}\xspace}
\newcommand{\CppIII}{\Cpp 2003\xspace}
\newcommand{\CppXI}{\Cpp 2011\xspace}
\newcommand{\CppXIV}{\Cpp 2014\xspace}
\newcommand{\opt}{{\ensuremath{_\mathit{opt}}}\xspace}
\newcommand{\shl}{<{<}}
\newcommand{\shr}{>{>}}
\newcommand{\dcr}{-{-}}
\newcommand{\exor}{\^{}}
\newcommand{\bigoh}[1]{\ensuremath{\mathscr{O}(#1)}}

% Make all tildes a little larger to avoid visual similarity with hyphens.
% FIXME: Remove \tilde in favour of \~.
\renewcommand{\tilde}{\textasciitilde}
\renewcommand{\~}{\textasciitilde}
\let\OldTextAsciiTilde\textasciitilde
\renewcommand{\textasciitilde}{\protect\raisebox{-0.17ex}{\larger\OldTextAsciiTilde}}

%%--------------------------------------------------
%% States and operators
\newcommand{\state}[2]{\tcode{#1}\ensuremath{_{#2}}}
\newcommand{\bitand}{\ensuremath{\, \mathsf{bitand} \,}}
\newcommand{\bitor}{\ensuremath{\, \mathsf{bitor} \,}}
\newcommand{\xor}{\ensuremath{\, \mathsf{xor} \,}}
\newcommand{\rightshift}{\ensuremath{\, \mathsf{rshift} \,}}
\newcommand{\leftshift}[1]{\ensuremath{\, \mathsf{lshift}_#1 \,}}

%% Notes and examples
\newcommand{\EnterBlock}[1]{[\,\textit{#1:}\xspace}
\newcommand{\ExitBlock}[1]{\textit{\,---\,end #1}\,]\xspace}
\newcommand{\enternote}{\EnterBlock{Note}}
\newcommand{\exitnote}{\ExitBlock{note}}
\newcommand{\enterexample}{\EnterBlock{Example}}
\newcommand{\exitexample}{\ExitBlock{example}}
%newer versions, legacy above!
\newcommand{\noteintro}[1]{[\,\textit{#1:}\space}
\newcommand{\noteoutro}[1]{\textit{\,---\,end #1}\,]}
\newenvironment{note}[1][Note]{\noteintro{#1}}{\noteoutro{note}\xspace}
\newenvironment{example}[1][Example]{\noteintro{#1}}{\noteoutro{example}\xspace}

%% Library function descriptions
\newcommand{\Fundescx}[1]{\textit{#1}\xspace}
\newcommand{\Fundesc}[1]{\Fundescx{#1:}}
\newcommand{\required}{\Fundesc{Required behavior}}
\newcommand{\requires}{\Fundesc{Requires}}
\newcommand{\effects}{\Fundesc{Effects}}
\newcommand{\postconditions}{\Fundesc{Postconditions}}
\newcommand{\postcondition}{\Fundesc{Postcondition}}
\newcommand{\preconditions}{\requires}
\newcommand{\precondition}{\requires}
\newcommand{\returns}{\Fundesc{Returns}}
\newcommand{\throws}{\Fundesc{Throws}}
\newcommand{\default}{\Fundesc{Default behavior}}
\newcommand{\complexity}{\Fundesc{Complexity}}
\newcommand{\remark}{\Fundesc{Remark}}
\newcommand{\remarks}{\Fundesc{Remarks}}
\newcommand{\realnote}{\Fundesc{Note}}
\newcommand{\realnotes}{\Fundesc{Notes}}
\newcommand{\errors}{\Fundesc{Error conditions}}
\newcommand{\sync}{\Fundesc{Synchronization}}
\newcommand{\implimits}{\Fundesc{Implementation limits}}
\newcommand{\replaceable}{\Fundesc{Replaceable}}
\newcommand{\returntype}{\Fundesc{Return type}}
\newcommand{\cvalue}{\Fundesc{Value}}
\newcommand{\ctype}{\Fundesc{Type}}
\newcommand{\ctypes}{\Fundesc{Types}}
\newcommand{\dtype}{\Fundesc{Default type}}
\newcommand{\ctemplate}{\Fundesc{Class template}}
\newcommand{\templalias}{\Fundesc{Alias template}}

%% Cross reference
\newcommand{\xref}{\textsc{See also:}\xspace}
\newcommand{\xsee}{\textsc{See:}\xspace}

%% NTBS, etc.
\newcommand{\NTS}[1]{\textsc{#1}\xspace}
\newcommand{\ntbs}{\NTS{ntbs}}
\newcommand{\ntmbs}{\NTS{ntmbs}}
\newcommand{\ntwcs}{\NTS{ntwcs}}
\newcommand{\ntcxvis}{\NTS{ntc16s}}
\newcommand{\ntcxxxiis}{\NTS{ntc32s}}

%% Code annotations
\newcommand{\EXPO}[1]{\textit{#1}}
\newcommand{\expos}{\EXPO{exposition only}}
\newcommand{\impdef}{\EXPO{implementation-defined}}
\newcommand{\impdefnc}{\EXPO{implementation-defined\nocorr}}
\newcommand{\impdefx}[1]{\indeximpldef{#1}\EXPO{implementation-defined}}
\newcommand{\notdef}{\EXPO{not defined}}

\newcommand{\UNSP}[1]{\textit{\texttt{#1}}}
\newcommand{\UNSPnc}[1]{\textit{\texttt{#1}\nocorr}}
\newcommand{\unspec}{\UNSP{unspecified}}
\newcommand{\unspecnc}{\UNSPnc{unspecified}}
\newcommand{\unspecbool}{\UNSP{unspecified-bool-type}}
\newcommand{\seebelow}{\UNSP{see below}}
\newcommand{\seebelownc}{\UNSPnc{see below}}
\newcommand{\unspecuniqtype}{\UNSP{unspecified unique type}}
\newcommand{\unspecalloctype}{\UNSP{unspecified allocator type}}

\newcommand{\EXPLICIT}{\textit{\texttt{EXPLICIT}}}

%% Manual insertion of italic corrections, for aligning in the presence
%% of the above annotations.
\newlength{\itcorrwidth}
\newlength{\itletterwidth}
\newcommand{\itcorr}[1][]{%
 \settowidth{\itcorrwidth}{\textit{x\/}}%
 \settowidth{\itletterwidth}{\textit{x\nocorr}}%
 \addtolength{\itcorrwidth}{-1\itletterwidth}%
 \makebox[#1\itcorrwidth]{}%
}

%% Double underscore
\newcommand{\ungap}{\kern.5pt}
\newcommand{\unun}{\_\ungap\_}
\newcommand{\xname}[1]{\tcode{\unun\ungap#1}}
\newcommand{\mname}[1]{\tcode{\unun\ungap#1\ungap\unun}}

%% Ranges
\newcommand{\Range}[4]{\tcode{#1#3,~\brk{}#4#2}\xspace}
\newcommand{\crange}[2]{\Range{[}{]}{#1}{#2}}
\newcommand{\brange}[2]{\Range{(}{]}{#1}{#2}}
\newcommand{\orange}[2]{\Range{(}{)}{#1}{#2}}
\newcommand{\range}[2]{\Range{[}{)}{#1}{#2}}

%% Change descriptions
\newcommand{\diffdef}[1]{\hfill\break\textbf{#1:}\xspace}
\newcommand{\change}{\diffdef{Change}}
\newcommand{\rationale}{\diffdef{Rationale}}
\newcommand{\effect}{\diffdef{Effect on original feature}}
\newcommand{\difficulty}{\diffdef{Difficulty of converting}}
\newcommand{\howwide}{\diffdef{How widely used}}

%% Miscellaneous
\newcommand{\uniquens}{\textrm{\textit{\textbf{unique}}}}
\newcommand{\stage}[1]{\item{\textbf{Stage #1:}}\xspace}
\newcommand{\doccite}[1]{\textit{#1}\xspace}
\newcommand{\cvqual}[1]{\textit{#1}}
\newcommand{\cv}{\cvqual{cv}}
\renewcommand{\emph}[1]{\textit{#1}\xspace}
\newcommand{\numconst}[1]{\textsl{#1}\xspace}
\newcommand{\logop}[1]{{\footnotesize #1}\xspace}

%%--------------------------------------------------
%% Environments for code listings.

% We use the 'listings' package, with some small customizations.  The
% most interesting customization: all TeX commands are available
% within comments.  Comments are set in italics, keywords and strings
% don't get special treatment.

\lstset{language=C++,
        basicstyle=\small\CodeStyle,
        keywordstyle=,
        stringstyle=,
        xleftmargin=1em,
        showstringspaces=false,
        commentstyle=\itshape\rmfamily,
        columns=flexible,
        keepspaces=true,
        texcl=true}

% Our usual abbreviation for 'listings'.  Comments are in 
% italics.  Arbitrary TeX commands can be used if they're 
% surrounded by @ signs.
\newcommand{\CodeBlockSetup}{
 \lstset{escapechar=@}
 \renewcommand{\tcode}[1]{\textup{\CodeStylex{##1}}}
 \renewcommand{\techterm}[1]{\textit{\CodeStylex{##1}}}
 \renewcommand{\term}[1]{\textit{##1}}
 \renewcommand{\grammarterm}[1]{\textit{##1}}
}

\lstnewenvironment{codeblock}{\CodeBlockSetup}{}

% A code block in which single-quotes are digit separators
% rather than character literals.
\lstnewenvironment{codeblockdigitsep}{
 \CodeBlockSetup
 \lstset{deletestring=[b]{'}}
}{}

% Permit use of '@' inside codeblock blocks (don't ask)
\makeatletter
\newcommand{\atsign}{@}
\makeatother

%%--------------------------------------------------
%% Indented text
\newenvironment{indented}
{\list{}{}\item\relax}
{\endlist}

%%--------------------------------------------------
%% Library item descriptions
\lstnewenvironment{itemdecl}
{
 \lstset{escapechar=@,
 xleftmargin=0em,
 aboveskip=2ex,
 belowskip=0ex	% leave this alone: it keeps these things out of the
				% footnote area
 }
}
{
}

\newenvironment{itemdescr}
{
 \begin{indented}}
{
 \end{indented}
}


%%--------------------------------------------------
%% Bnf environments
\newlength{\BnfIndent}
\setlength{\BnfIndent}{\leftmargini}
\newlength{\BnfInc}
\setlength{\BnfInc}{\BnfIndent}
\newlength{\BnfRest}
\setlength{\BnfRest}{2\BnfIndent}
\newcommand{\BnfNontermshape}{\small\rmfamily\itshape}
\newcommand{\BnfTermshape}{\small\ttfamily\upshape}
\newcommand{\nonterminal}[1]{{\BnfNontermshape #1}}

\newenvironment{bnfbase}
 {
 \newcommand{\nontermdef}[1]{\nonterminal{##1}\indexgrammar{\idxgram{##1}}:}
 \newcommand{\terminal}[1]{{\BnfTermshape ##1}\xspace}
 \newcommand{\descr}[1]{\normalfont{##1}}
 \newcommand{\bnfindentfirst}{\BnfIndent}
 \newcommand{\bnfindentinc}{\BnfInc}
 \newcommand{\bnfindentrest}{\BnfRest}
 \begin{minipage}{.9\hsize}
 \newcommand{\br}{\hfill\\}
 \frenchspacing
 }
 {
 \nonfrenchspacing
 \end{minipage}
 }

\newenvironment{BnfTabBase}[1]
{
 \begin{bnfbase}
 #1
 \begin{indented}
 \begin{tabbing}
 \hspace*{\bnfindentfirst}\=\hspace{\bnfindentinc}\=\hspace{.6in}\=\hspace{.6in}\=\hspace{.6in}\=\hspace{.6in}\=\hspace{.6in}\=\hspace{.6in}\=\hspace{.6in}\=\hspace{.6in}\=\hspace{.6in}\=\hspace{.6in}\=\kill}
{
 \end{tabbing}
 \end{indented}
 \end{bnfbase}
}

\newenvironment{bnfkeywordtab}
{
 \begin{BnfTabBase}{\BnfTermshape}
}
{
 \end{BnfTabBase}
}

\newenvironment{bnftab}
{
 \begin{BnfTabBase}{\BnfNontermshape}
}
{
 \end{BnfTabBase}
}

\newenvironment{simplebnf}
{
 \begin{bnfbase}
 \BnfNontermshape
 \begin{indented}
}
{
 \end{indented}
 \end{bnfbase}
}

\newenvironment{bnf}
{
 \begin{bnfbase}
 \list{}
	{
	\setlength{\leftmargin}{\bnfindentrest}
	\setlength{\listparindent}{-\bnfindentinc}
	\setlength{\itemindent}{\listparindent}
	}
 \BnfNontermshape
 \item\relax
}
{
 \endlist
 \end{bnfbase}
}

% non-copied versions of bnf environments
\newenvironment{ncbnftab}
{
 \begin{bnftab}
}
{
 \end{bnftab}
}

\newenvironment{ncsimplebnf}
{
 \begin{simplebnf}
}
{
 \end{simplebnf}
}

\newenvironment{ncbnf}
{
 \begin{bnf}
}
{
 \end{bnf}
}

%%--------------------------------------------------
%% Drawing environment
%
% usage: \begin{drawing}{UNITLENGTH}{WIDTH}{HEIGHT}{CAPTION}
\newenvironment{drawing}[4]
{
\newcommand{\mycaption}{#4}
\begin{figure}[h]
\setlength{\unitlength}{#1}
\begin{center}
\begin{picture}(#2,#3)\thicklines
}
{
\end{picture}
\end{center}
\caption{\mycaption}
\end{figure}
}

%%--------------------------------------------------
%% Environment for imported graphics
% usage: \begin{importgraphic}{CAPTION}{TAG}{FILE}

\newenvironment{importgraphic}[3]
{%
\newcommand{\cptn}{#1}
\newcommand{\lbl}{#2}
\begin{figure}[htp]\centering%
\includegraphics[scale=.35]{#3}
}
{
\caption{\cptn}\label{\lbl}%
\end{figure}}

%% enumeration display overrides
% enumerate with lowercase letters
\newenvironment{enumeratea}
{
 \renewcommand{\labelenumi}{\alph{enumi})}
 \begin{enumerate}
}
{
 \end{enumerate}
}

% enumerate with arabic numbers
\newenvironment{enumeraten}
{
 \renewcommand{\labelenumi}{\arabic{enumi})}
 \begin{enumerate}
}
{
 \end{enumerate}
}

%%--------------------------------------------------
%% Definitions section
% usage: \definition{name}{xref}
%\newcommand{\definition}[2]{\rSec2[#2]{#1}}
% for ISO format, use:
\newcommand{\definition}[2]{%
\subsection[#1]{\hfill[#2]}\vspace{-.3\onelineskip}\label{#2}\textbf{#1}\\%
}
\newcommand{\definitionx}[2]{%
\subsubsection[#1]{\hfill[#2]}\vspace{-.3\onelineskip}\label{#2}\textbf{#1}\\%
}
\newcommand{\defncontext}[1]{\textlangle#1\textrangle}
 
 %% adopted from standard's layout.tex
 \newcounter{Paras}
\counterwithin{Paras}{chapter}
\counterwithin{Paras}{section}
\counterwithin{Paras}{subsection}
\counterwithin{Paras}{subsubsection}
\counterwithin{Paras}{paragraph}
\counterwithin{Paras}{subparagraph}

 \makeatletter
\def\pnum{\addtocounter{Paras}{1}\noindent\llap{{%
  \scriptsize\raisebox{.7ex}{\arabic{Paras}}}\hspace{\@totalleftmargin}\quad}}
\makeatother

%% PS: add some helpers for coloring, assumes \usepackage{color}
%% should no longer be used, since we have those macros already!
\newcommand{\del}[1]{\removed{#1}}
\newcommand{\ins}[1]{\added{#1}}

\newenvironment{insrt}{\begin{addedblock}}{\end{addedblock}}


\setsecnumdepth{subsection}

\begin{document}
\maketitle
\begin{tabular}[t]{|l|l|}\hline 
Document Number: \papernumber & (N2065 done right?)\\\hline
Date: & \paperdate \\\hline
Project: & Programming Language C++\\\hline 
Audience: & LWG/LEWG\\\hline
\end{tabular}

\chapter{History}
Streams have been the oldest part of the C++ standard library and especially strstreams that can use pre-allocated buffers have been deprecated for a long time now, waiting for a replacement. p0407 and p0408 provide the efficient access to the underlying buffer for stringstreams that strstream provided solving half of the problem that strstreams provide a solution for. The other half is using a fixed size pre-allocated buffer, e.g., allocated on the stack, that is used as the stream buffers internal storage.

A combination of external-fixed and internal-growing buffer allocation that strstreambuf provides is IMHO a doomed approach and very hard to use right.

There had been a proposal for the pre-allocated external memory buffer streams in N2065 but that went nowhere. Today, with \tcode{span<T>} we actually have a library type representing such buffers views we can use for specifying (and implementing) such streams. They can be used in areas where dynamic (re-)allocation of stringstreams is not acceptable but the burden of caring for a pre-existing buffer during the lifetime of the stream is manageable. 

\chapter{Introduction}
This paper proposes a class template \tcode{basic_spanbuf} and the corresponding stream class templates to enable the use of streams on externally provided memory buffers. No ownership or re-allocation support is given. For those features we have string-based streams.

\chapter{Acknowledgements}
\begin{itemize}
\item Thanks to those ISO C++ meeting members attending the Oulu meeting encouring me to write this proposal. I believe Neil and Pablo have been among them, but can't remember who else.
\item Thanks go to Jonathan Wakely who pointed the problem of \tcode{strstream} out to me and to Neil Macintosh to provide the span library type specification.
\item Thanks to Felix Morgner for proofreading.
\end{itemize}

\chapter{Motivation}
To finally get rid of the deprecated \tcode{strstream} in the C++ standard we need a replacement. p0407/p0408 provide one for one half of the needs for \tcode{strstream}. This paper provides one for the second half: fixed sized buffers. 

\begin{example} reading input from a fixed pre-arranged character buffer:
\begin{codeblock}
char input[] = "10 20 30";
ispanstream is{span<char>{input}};
int i;
is >> i;
ASSERT_EQUAL(10,i);
is >> i ;
ASSERT_EQUAL(20,i);
is >> i;
ASSERT_EQUAL(30,i);
is >>i;
ASSERT(!is);
\end{codeblock}
\end{example}
\begin{example} writing to a fixed pre-arranged character buffer:
\begin{codeblock}
char  output[30]{}; // zero-initialize array
ospanstream os{span<char>{output}};
os << 10 << 20 << 30 ;
auto const sp = os.span();
ASSERT_EQUAL(6,sp.size());
ASSERT_EQUAL("102030",std::string(sp.data(),sp.size()));
ASSERT_EQUAL(static_cast<void*>(output),sp.data()); // no copying of underlying data!
ASSERT_EQUAL("102030",output); // initialization guaranteed NUL termination
\end{codeblock}
\end{example}

\chapter{Impact on the Standard}
This is an extension to the standard library to enable deletion of the deprecated \tcode{strstream} classes by providing \tcode{basic_spanbuf}, \tcode{basic_spanstream}, \tcode{basic_ispanstream}, and \tcode{basic_ospanstream} class templates that take an object of type \tcode{span<charT>} which provides an external buffer to be used by the stream. 

\chapter{Design Decisions}
\section{General Principles}
\section{Open Issues to be Discussed by LEWG / LWG}
\begin{itemize}
\item Should arbitrary types as template arguments to \tcode{span} be allowed to provide the underlying buffer by using the \tcode{byte} sequence representation \tcode{span} provides. (I do not think so, but someone might have a usecase.)
\item Should the \tcode{basic_spanbuf} be copy-able? It doesn't own any resources, so copying like with handles or \tcode{span} might be fine.
\end{itemize}

\chapter{Technical Specifications}
Insert a new section 27.x in chapter 27 after section 27.8 [string.streams]

\section{27.x Span-based Streams [span.streams]}
This section introduces a stream interface for user-provided fixed-size buffers. 
\subsection{27.x.1 Overview [span.streams.overview]}
The header \tcode{<spanstream>} defines four class templates and eight types that associate stream buffers with objects of class \tcode{span} as described in [span].

\paragraph{Header \tcode{<spanstream>} synobsis}

\begin{codeblock}
namespace std {
namespace experimental {
  template <class charT, class traits = char_traits<charT> >
    class basic_spanbuf;
  typedef basic_spanbuf<char>     spanbuf;
  typedef basic_spanbuf<wchar_t> wspanbuf;
  template <class charT, class traits = char_traits<charT> >
    class basic_ispanstream;
  typedef basic_ispanstream<char>     ispanstream;
  typedef basic_ispanstream<wchar_t> wispanstream;
  template <class charT, class traits = char_traits<charT> >
    class basic_ospanstream;
  typedef basic_ospanstream<char>     ospanstream;
  typedef basic_ospanstream<wchar_t> wospanstream;
  template <class charT, class traits = char_traits<charT> >
    class basic_spanstream;
  typedef basic_spanstream<char>     spanstream;
  typedef basic_spanstream<wchar_t> wspanstream;
}}
\end{codeblock}
\section{27.x.2 Class template \tcode{basic_spanbuf} [spanbuf]}
%\rSec2[spanbuf]{Class template \tcode{basic_spanbuf}}
%\indexlibrary{\idxcode{basic_spanbuf}}%
\begin{codeblock}
namespace std {
  template <class charT, class traits = char_traits<charT> >
  class basic_spanbuf
    : public basic_streambuf<charT, traits> {
  public:
    using char_type      = charT;
    using int_type       = typename traits::int_type;
    using pos_type       = typename traits::pos_type;
    using off_type       = typename traits::off_type;
    using traits_type    = traits;

    // \ref{spanbuf.cons}, constructors:
    template <ptrdiff_t Extent>
    explicit basic_spanbuf(
      span<charT, Extent> span,
      ios_base::openmode which = ios_base::in | ios_base::out);
    basic_spanbuf(const basic_spanbuf& rhs) = delete;
    basic_spanbuf(basic_spanbuf&& rhs) noexcept;

    // \ref{spanbuf.assign}, assign and swap:
    basic_spanbuf& operator=(const basic_spanbuf& rhs) = delete;
    basic_spanbuf& operator=(basic_spanbuf&& rhs) noexcept;
    void swap(basic_spanbuf& rhs) noexcept;

    // \ref{spanbuf.members}, get and set:
    span<charT> span() const noexcept;
    void span(span<charT> s) noexcept;

  protected:
    // \ref{spanbuf.virtuals}, overridden virtual functions:
    int_type underflow() override;
    int_type pbackfail(int_type c = traits::eof()) override;
    int_type overflow (int_type c = traits::eof()) override;
    basic_streambuf<charT, traits>* setbuf(charT*, streamsize) override;

    pos_type seekoff(off_type off, ios_base::seekdir way,
                     ios_base::openmode which
                      = ios_base::in | ios_base::out) override;
    pos_type seekpos(pos_type sp,
                     ios_base::openmode which
                      = ios_base::in | ios_base::out) override;

  private:
    ios_base::openmode mode;  // \expos
  };

  template <class charT, class traits>
    void swap(basic_spanbuf<charT, traits>& x,
              basic_spanbuf<charT, traits>& y) noexcept;
}
\end{codeblock}

\pnum
The class
\tcode{basic_spanbuf}
is derived from
\tcode{basic_streambuf}
to associate possibly the input sequence and possibly
the output sequence with a sequence of arbitrary
\term{characters}.
The sequence is provided by an object of class
\tcode{span<charT>}.

\pnum
For the sake of exposition, the maintained data is presented here as:
\begin{itemize}
\item
\tcode{ios_base::openmode mode},
has
\tcode{in}
set if the input sequence can be read, and
\tcode{out}
set if the output sequence can be written.
\end{itemize}
%%%

\section{27.x.2.1 \tcode{basic_spanbuf} constructors [spanbuf.cons]}
%\rSec3[spanbuf.cons]{\tcode{basic_spanbuf}  constructors}

\indexlibrary{\idxcode{basic_spanbuf}!constructor}%
\begin{itemdecl}
template <ptrdiff_t Extent>
explicit basic_spanbuf(
  basic_span<charT, Extent> s,
  ios_base::openmode which = ios_base::in | ios_base::out);
\end{itemdecl}

\begin{itemdescr}
\pnum
\effects
Constructs an object of class
\tcode{basic_spanbuf},
initializing the base class with
\tcode{basic_streambuf()}~(\ref{streambuf.cons}), and initializing
\tcode{mode}
with \tcode{which}. 
Initializes the internal pointers as if calling \tcode{span(s)}.
\end{itemdescr}

\indexlibrary{\idxcode{basic_spanbuf}!constructor}%
\begin{itemdecl}
basic_spanbuf(basic_spanbuf&& rhs) noexcept;
\end{itemdecl}

\begin{itemdescr}
\pnum
\effects Move constructs from the rvalue \tcode{rhs}. 
Both \tcode{basic_spanbuf} objects share the same underlying \tcode{span}.
The sequence pointers in 
\tcode{*this}
(\tcode{eback()}, \tcode{gptr()}, \tcode{egptr()},
\tcode{pbase()}, \tcode{pptr()}, \tcode{epptr()}) obtain
the values which \tcode{rhs} had. 
%It
%is
%\impldef{whether sequence pointers are copied by \tcode{basic_spanbuf} move
%constructor} whether the sequence pointers in \tcode{*this}
%(\tcode{eback()}, \tcode{gptr()}, \tcode{egptr()},
%\tcode{pbase()}, \tcode{pptr()}, \tcode{epptr()}) obtain
%the values which \tcode{rhs} had. Whether they do or not, \tcode{*this}
%and \tcode{rhs} reference the same \tcode{span} after the
%construction. 
The openmode, locale and any other state of \tcode{rhs} is
also copied.

\pnum
\postconditions Let \tcode{rhs_p} refer to the state of
\tcode{rhs} just prior to this construction.

\begin{itemize}
\item \tcode{span() == rhs_p.span()}
\item \tcode{eback() == rhs_p.eback()}
\item \tcode{gptr() == rhs_p.gptr()}
\item \tcode{egptr() == rhs_p.egptr()}
\item \tcode{pbase() == rhs_p.pbase()}
\item \tcode{pptr() == rhs_p.pptr()}
\item \tcode{epptr() == rhs_p.epptr()}
\end{itemize}
\end{itemdescr}


\subsection{27.x.2.2 Assign and swap [spanbuf.assign]}

%\rSec3[spanbuf.assign]{Assign and swap}
%\indexlibrarymember{operator=}{basic_spanbuf}%
\begin{itemdecl}
basic_spanbuf& operator=(basic_spanbuf&& rhs) noexcept;
\end{itemdecl}

\begin{itemdescr}
\pnum
\effects After the move assignment \tcode{*this} has the observable state it would
have had if it had been move constructed from \tcode{rhs} (see~\ref{spanbuf.cons}).

\pnum
\returns \tcode{*this}.
\end{itemdescr}

%\indexlibrarymember{swap}{basic_spanbuf}%
\begin{itemdecl}
void swap(basic_spanbuf& rhs) noexcept;
\end{itemdecl}

\begin{itemdescr}
\pnum
\effects Exchanges the state of \tcode{*this}
and \tcode{rhs}.
\end{itemdescr}

%\indexlibrarymember{swap}{basic_spanbuf}%
\begin{itemdecl}
template <class charT, class traits, class Allocator>
  void swap(basic_spanbuf<charT, traits>& x,
            basic_spanbuf<charT, traits>& y) noexcept;
\end{itemdecl}

\begin{itemdescr}
\pnum
\effects As if by \tcode{x.swap(y)}.
\end{itemdescr}


\subsection{27.x.2.3 Member functions [spanbuf.members]}
%\rSec3[spanbuf.members]{Member functions}

%\indexlibrarymember{span}{basic_spanbuf}%
\begin{itemdecl}
span<charT> span() const;
\end{itemdecl}

\begin{itemdescr}
\pnum
\returns
A
\tcode{span}
object representing the  
\tcode{basic_spanbuf}
underlying character sequence.
If the \tcode{basic_spanbuf} was created only in output mode, the resultant
\tcode{span} represents the character sequence in the range
\range{pbase()}{pptr()}, otherwise in the range
\range{eback()}{egptr()}. 
\begin{note}
In constrast to \tcode{basic_stringbuf} the underlying sequence can never grow and will not be owned. An owning copy can be obtained by converting the result to \tcode{basic_string<charT>}.
\end{note}


\end{itemdescr}

%\indexlibrarymember{str}{basic_spanbuf}%
\begin{itemdecl}
template<ptrdiff_t Extent>
void span(span<charT,Extent> s);
\end{itemdecl}

\begin{itemdescr}
\pnum
\effects
Initializes the \tcode{basic_spanbuf} underlying character
sequence with \tcode{s} and initializes the input and output sequences according to \tcode{mode}.

\pnum
\postconditions If \tcode{mode \& ios_base::out} is true, \tcode{pbase()} points to the
first underlying character and \tcode{epptr()} \tcode{>= pbase() + s.size()} holds; in
addition, if \tcode{mode \& ios_base::ate} is true,
\tcode{pptr() == pbase() + s.size()}
holds, otherwise \tcode{pptr() == pbase()} is true. If \tcode{mode \& ios_base::in} is
true, \tcode{eback()} points to the first underlying character, and both \tcode{gptr()
== eback()} and \tcode{egptr() == eback() + s.size()} hold.

\begin{note}
Using append mode does not make sense for \tcode{span}-based streams.
\end{note}

\end{itemdescr}

\subsection{27.x.2.4 Overridden virtual functions [spanbuf.virtuals]}
%\rSec3[spanbuf.virtuals]{Overridden virtual functions}
\pnum
\begin{note}
Since the underlying buffer is of fixed size, neither \tcode{overflow}, \tcode{underflow} or \tcode{pbackfail} can provide useful behavior.
\end{note}

%\indexlibrarymember{underflow}{basic_spanbuf}%
\begin{itemdecl}
int_type underflow() override;
\end{itemdecl}

\begin{itemdescr}
\pnum
\returns
%If the input sequence has a read position available,
%returns
%\tcode{traits::to_int_type(*gptr())}.
%Otherwise, returns
\tcode{traits::eof()}.
%Any character in the underlying buffer which has been initialized is considered
%to be part of the input sequence. 
\end{itemdescr}

%\indexlibrarymember{pbackfail}{basic_spanbuf}%
\begin{itemdecl}
int_type pbackfail(int_type c = traits::eof()) override;
\end{itemdecl}

\begin{itemdescr}
\pnum
\returns
\tcode{traits::eof()}.
\end{itemdescr}

%\indexlibrarymember{overflow}{basic_spanbuf}%
\begin{itemdecl}
int_type overflow(int_type c = traits::eof()) override;
\end{itemdecl}

\begin{itemdescr}
\pnum
\returns
\tcode{traits::eof()}.

\end{itemdescr}

%\indexlibrarymember{seekoff}{basic_spanbuf}%
\begin{itemdecl}
pos_type seekoff(off_type off, ios_base::seekdir way,
                 ios_base::openmode which
                   = ios_base::in | ios_base::out) override;
\end{itemdecl}

\begin{itemdescr}
\pnum
\effects
Alters the stream position within one of the
controlled sequences, if possible, as indicated in Table~\ref{tab:iostreams.seekoff.positioning}.

%\begin{libtab2}{\tcode{seekoff} positioning}{tab:iostreams.seekoff.positioning}
%{p{2.5in}l}{Conditions}{Result}
%\tcode{(which \& ios_base\colcol{}in)}\tcode{ == ios_base::in}  &
% positions the input sequence \\ \rowsep
%\tcode{(which \& ios_base\colcol{}out)}\tcode{ == ios_base::out}  &
% positions the output sequence  \\ \rowsep
%\tcode{(which \& (ios_base::in |}\br
%\tcode{ios_base::out)) ==}\br
%\tcode{(ios_base::in) |}\br
%\tcode{ios_base::out))}\br
%and \tcode{way ==} either\br
%\tcode{ios_base::beg} or\br
%\tcode{ios_base::end}     &
% positions both the input and the output sequences  \\ \rowsep
%Otherwise &
% the positioning operation fails. \\
%\end{libtab2}

\pnum
For a sequence to be positioned, if its next pointer
(either
\tcode{gptr()}
or
\tcode{pptr()})
is a null pointer and the new offset \tcode{newoff} is nonzero, the positioning
operation fails. Otherwise, the function determines \tcode{newoff} as indicated in
Table~\ref{tab:iostreams.newoff.values}.

%\begin{libtab2}{\tcode{newoff} values}{tab:iostreams.newoff.values}
%{lp{2.0in}}{Condition}{\tcode{newoff} Value}
%\tcode{way == ios_base::beg}  &
% 0  \\ \rowsep
%\tcode{way == ios_base::cur}  &
% the next pointer minus the beginning pointer (\tcode{xnext - xbeg}). \\ \rowsep
%\tcode{way == ios_base::end}  &
% the high mark pointer minus the beginning pointer (\tcode{high_mark - xbeg}).   \\
%\end{libtab2}

\pnum
If
\tcode{(newoff + off) < 0},
or if \tcode{newoff + off} refers to an uninitialized
character outside the span (as defined in~\ref{spanbuf.members} paragraph 1),
the positioning operation fails.
Otherwise, the function assigns
\tcode{xbeg + newoff + off}
to the next pointer \tcode{xnext}.

\pnum
\returns
\tcode{pos_type(newoff)},
constructed from the resultant offset \tcode{newoff}
(of type
\tcode{off_type}),
that stores the resultant stream position, if possible.
If the positioning operation fails, or
if the constructed object cannot represent the resultant stream position,
the return value is
\tcode{pos_type(off_type(-1))}.
\end{itemdescr}

%\indexlibrarymember{seekpos}{basic_spanbuf}%
\begin{itemdecl}
pos_type seekpos(pos_type sp,
                 ios_base::openmode which
                   = ios_base::in | ios_base::out) override;
\end{itemdecl}

\begin{itemdescr}
\pnum
\effects
Equivalent to \tcode{seekoff(off_type(sp), ios_base::beg, which)}.

\pnum
\returns
\tcode{sp}
to indicate success, or
\tcode{pos_type(off_type(-1))}
to indicate failure.
\end{itemdescr}

%\indexlibrarymember{setbuf}{basic_streambuf}%
\begin{itemdecl}
basic_streambuf<charT, traits>* setbuf(charT* s, streamsize n);
\end{itemdecl}

\begin{itemdescr}
\pnum
\effects
If \tcode{s} and \tcode{n} denote a non-empty span
\tcode{this->span(span<charT>(s,n));}

\pnum
\returns
\tcode{this}.
\end{itemdescr}

%\rSec2[ispanstream]{Class template \tcode{basic_ispanstream}}
\section{27.x.3 Class template \tcode{basic_ispanstream} [ispanstream] }

%\indexlibrary{\idxcode{basic_ispanstream}}%
\begin{codeblock}
namespace std {
  template <class charT, class traits = char_traits<charT>>
  class basic_ispanstream
    : public basic_istream<charT, traits> {
  public:
    using char_type      = charT;
    using int_type       = typename traits::int_type;
    using pos_type       = typename traits::pos_type;
    using off_type       = typename traits::off_type;
    using traits_type    = traits;

    // \ref{ispanstream.cons}, constructors:
    template <ptrdiff_t Extent>
    explicit basic_ispanstream(
      span<charT, Extent> span,
      ios_base::openmode which = ios_base::in);
    basic_ispanstream(const basic_ispanstream& rhs) = delete;
    basic_ispanstream(basic_ispanstream&& rhs) noexcept;

    // \ref{ispanstream.assign}, assign and swap:
    basic_ispanstream& operator=(const basic_ispanstream& rhs) = delete;
    basic_ispanstream& operator=(basic_ispanstream&& rhs) noexcept;
    void swap(basic_ispanstream& rhs) noexcept;

    // \ref{ispanstream.members}, members:
    basic_spanbuf<charT, traits>* rdbuf() const noexcept;

    span<charT> span() const noexcept;
	template<ptrdiff_t Extent>
    void span(span<charT> s) noexcept;
  private:
    basic_spanbuf<charT, traits> sb; // \expos
  };

  template <class charT, class traits>
    void swap(basic_ispanstream<charT, traits>& x,
              basic_ispanstream<charT, traits>& y) noexcept;
}
\end{codeblock}

\pnum
The class
\tcode{basic_ispanstream<charT, traits>}
supports reading objects of class
\tcode{span<charT, traits>}.
It uses a
\tcode{basic_spanbuf<charT, traits>}
object to control the associated span.
For the sake of exposition, the maintained data is presented here as:
\begin{itemize}
\item
\tcode{sb}, the \tcode{spanbuf} object.
\end{itemize}

%\rSec3[ispanstream.cons]{\tcode{basic_ispanstream} constructors}
\subsection{27.x.3.1 \tcode{basic_ispanstream} constructors [ispanstream.cons]}
\label{ispanstream.cons}

%\indexlibrary{\idxcode{basic_ispanstream}!constructor}%
\begin{itemdecl}
template <ptrdiff_t Extent>
explicit basic_ispanstream(
  span<charT, Extent> span,
  ios_base::openmode which = ios_base::in);
\end{itemdecl}

\begin{itemdescr}
\pnum
\effects
Constructs an object of class
\tcode{basic_ispanstream<charT, traits>},
initializing the base class with
\tcode{basic_istream(\&sb)}
and initializing \tcode{sb} with
\tcode{basic_spanbuf<charT, traits>{span, which | ios_base::in})}~(\ref{spanbuf.cons}).
\end{itemdescr}

%\indexlibrary{\idxcode{basic_ispanstream}!constructor}%
\begin{itemdecl}
basic_ispanstream(basic_ispanstream&& rhs);
\end{itemdecl}

\begin{itemdescr}
\pnum
\effects Move constructs from the rvalue \tcode{rhs}. This
is accomplished by move constructing the base class, and the contained
\tcode{basic_spanbuf}.
Next \tcode{basic_istream<charT, traits>::set_rdbuf(\&sb)} is called to
install the contained \tcode{basic_spanbuf}.
\end{itemdescr}

%\rSec3[ispanstream.assign]{Assign and swap}
\subsection{27.x.3.2 Assign and swap [ispanstream.assign]}
\label{ispanstream.assign}

%\indexlibrarymember{operator=}{basic_ispanstream}%
\begin{itemdecl}
basic_ispanstream& operator=(basic_ispanstream&& rhs);
\end{itemdecl}

\begin{itemdescr}
\pnum
\effects Move assigns the base and members of \tcode{*this} from the base and corresponding
members of \tcode{rhs}.

\pnum
\returns \tcode{*this}.
\end{itemdescr}

%\indexlibrarymember{swap}{basic_ispanstream}%
\begin{itemdecl}
void swap(basic_ispanstream& rhs);
\end{itemdecl}

\begin{itemdescr}
\pnum
\effects Exchanges the state of \tcode{*this} and
\tcode{rhs} by calling
\tcode{basic_istream<charT, traits>::swap(rhs)} and
\tcode{sb.swap(rhs.sb)}.
\end{itemdescr}


%\indexlibrarymember{swap}{basic_ispanstream}%
\begin{itemdecl}
template <class charT, class traits, class Allocator>
  void swap(basic_ispanstream<charT, traits, Allocator>& x,
            basic_ispanstream<charT, traits, Allocator>& y);
\end{itemdecl}

\begin{itemdescr}
\pnum
\effects As if by \tcode{x.swap(y)}.
\end{itemdescr}

%\rSec3[ispanstream.members]{Member functions}
\subsection{27.x.3.3 Member functions [ispanstream.members]}
\label{ispanstream.members}

%\indexlibrarymember{rdbuf}{basic_ispanstream}%
\begin{itemdecl}
basic_spanbuf<charT>* rdbuf() const noexcept;
\end{itemdecl}

\begin{itemdescr}
\pnum
\returns
\tcode{const_cast<basic_spanbuf<charT>*>(\&sb)}.
\end{itemdescr}

%\indexlibrarymember{str}{basic_ispanstream}%
\begin{itemdecl}
span<charT> span() const noexcept;
\end{itemdecl}

\begin{itemdescr}
\pnum
\returns
\tcode{rdbuf()->span()}.
\end{itemdescr}

%\indexlibrarymember{str}{basic_ispanstream}%
\begin{itemdecl}
template<ptrdiff_t Extent>
void span(span<charT, Extent> s) noexcept;
\end{itemdecl}

\begin{itemdescr}
\pnum
\effects
Calls
\tcode{rdbuf()->span(s)}.
\end{itemdescr}

%\rSec2[ospanstream]{Class template \tcode{basic_ospanstream}}
\section{27.x.4 Class template \tcode{basic_ospanstream} [ospanstream] }

%\indexlibrary{\idxcode{basic_ospanstream}}%
\begin{codeblock}
namespace std {
  template <class charT, class traits = char_traits<charT>>
  class basic_ospanstream
    : public basic_ostream<charT, traits> {
  public:
    using char_type      = charT;
    using int_type       = typename traits::int_type;
    using pos_type       = typename traits::pos_type;
    using off_type       = typename traits::off_type;
    using traits_type    = traits;

    // \ref{ospanstream.cons}, constructors:
    template <ptrdiff_t Extent>
    explicit basic_ospanstream(
      span<charT, Extent> span,
      ios_base::openmode which = ios_base::out);
    basic_ospanstream(const basic_ospanstream& rhs) = delete;
    basic_ospanstream(basic_ospanstream&& rhs) noexcept;

    // \ref{ospanstream.assign}, assign and swap:
    basic_ospanstream& operator=(const basic_ospanstream& rhs) = delete;
    basic_ospanstream& operator=(basic_ospanstream&& rhs) noexcept;
    void swap(basic_ospanstream& rhs) noexcept;

    // \ref{ospanstream.members}, members:
    basic_spanbuf<charT, traits>* rdbuf() const noexcept;

    span<charT> span() const noexcept;
	template<ptrdiff_t Extent>
    void span(span<charT> s) noexcept;
  private:
    basic_spanbuf<charT, traits> sb; // \expos
  };

  template <class charT, class traits>
    void swap(basic_ospanstream<charT, traits>& x,
              basic_ospanstream<charT, traits>& y) noexcept;
}
\end{codeblock}

\pnum
The class
\tcode{basic_ospanstream<charT, traits>}
supports writing to objects of class
\tcode{span<charT, traits>}.
It uses a
\tcode{basic_spanbuf<charT, traits>}
object to control the associated span.
For the sake of exposition, the maintained data is presented here as:
\begin{itemize}
\item
\tcode{sb}, the \tcode{spanbuf} object.
\end{itemize}

%\rSec3[ospanstream.cons]{\tcode{basic_ospanstream} constructors}
\subsection{27.x.4.1 \tcode{basic_ospanstream} constructors [ospanstream.cons]}
\label{ospanstream.cons}

%\indexlibrary{\idxcode{basic_ospanstream}!constructor}%
\begin{itemdecl}
template <ptrdiff_t Extent>
explicit basic_ospanstream(
  span<charT, Extent> span,
  ios_base::openmode which = ios_base::out);
\end{itemdecl}

\begin{itemdescr}
\pnum
\effects
Constructs an object of class
\tcode{basic_ospanstream<charT, traits>},
initializing the base class with
\tcode{basic_ostream(\&sb)}
and initializing \tcode{sb} with
\tcode{basic_spanbuf<charT, traits>{span, which | ios_base::out})}~(\ref{spanbuf.cons}).
\end{itemdescr}

%\indexlibrary{\idxcode{basic_ospanstream}!constructor}%
\begin{itemdecl}
basic_ospanstream(basic_ospanstream&& rhs) noexcept;
\end{itemdecl}

\begin{itemdescr}
\pnum
\effects Move constructs from the rvalue \tcode{rhs}. This
is accomplished by move constructing the base class, and the contained
\tcode{basic_spanbuf}.
Next \tcode{basic_ostream<charT, traits>::set_rdbuf(\&sb)} is called to
install the contained \tcode{basic_spanbuf}.
\end{itemdescr}

%\rSec3[ospanstream.assign]{Assign and swap}
\subsection{27.x.4.2 Assign and swap [ospanstream.assign]}
\label{ospanstream.assign}

%\indexlibrarymember{operator=}{basic_ospanstream}%
\begin{itemdecl}
basic_ospanstream& operator=(basic_ospanstream&& rhs) noexcept;
\end{itemdecl}

\begin{itemdescr}
\pnum
\effects Move assigns the base and members of \tcode{*this} from the base and corresponding
members of \tcode{rhs}.

\pnum
\returns \tcode{*this}.
\end{itemdescr}

%\indexlibrarymember{swap}{basic_ospanstream}%
\begin{itemdecl}
void swap(basic_ospanstream& rhs) noexcept;
\end{itemdecl}

\begin{itemdescr}
\pnum
\effects Exchanges the state of \tcode{*this} and
\tcode{rhs} by calling
\tcode{basic_ostream<charT, traits>::swap(rhs)} and
\tcode{sb.swap(rhs.sb)}.
\end{itemdescr}


%\indexlibrarymember{swap}{basic_ospanstream}%
\begin{itemdecl}
template <class charT, class traits, class Allocator>
  void swap(basic_ospanstream<charT, traits, Allocator>& x,
            basic_ospanstream<charT, traits, Allocator>& y) noexcept;
\end{itemdecl}

\begin{itemdescr}
\pnum
\effects As if by \tcode{x.swap(y)}.
\end{itemdescr}

%\rSec3[ospanstream.members]{Member functions}
\subsection{27.x.4.3 Member functions [ospanstream.members]}
\label{ospanstream.members}

%\indexlibrarymember{rdbuf}{basic_ospanstream}%
\begin{itemdecl}
basic_spanbuf<charT>* rdbuf() const noexcept;
\end{itemdecl}

\begin{itemdescr}
\pnum
\returns
\tcode{const_cast<basic_spanbuf<charT>*>(\&sb)}.
\end{itemdescr}

%\indexlibrarymember{str}{basic_ospanstream}%
\begin{itemdecl}
span<charT> span() const noexcept;
\end{itemdecl}

\begin{itemdescr}
\pnum
\returns
\tcode{rdbuf()->span()}.
\end{itemdescr}

%\indexlibrarymember{str}{basic_ospanstream}%
\begin{itemdecl}
template<ptrdiff_t Extent>
void span(span<charT, Extent> s) noexcept;
\end{itemdecl}

\begin{itemdescr}
\pnum
\effects
Calls
\tcode{rdbuf()->span(s)}.
\end{itemdescr}

%\rSec2[spanstream]{Class template \tcode{basic_spanstream}}
\section{27.x.5 Class template \tcode{basic_spanstream} [spanstream] }

%\indexlibrary{\idxcode{basic_spanstream}}%
\begin{codeblock}
namespace std {
  template <class charT, class traits = char_traits<charT>>
  class basic_spanstream
    : public basic_iostream<charT, traits> {
  public:
    using char_type      = charT;
    using int_type       = typename traits::int_type;
    using pos_type       = typename traits::pos_type;
    using off_type       = typename traits::off_type;
    using traits_type    = traits;

    // \ref{spanstream.cons}, constructors:
    template <ptrdiff_t Extent>
    explicit basic_spanstream(
      span<charT, Extent> span,
      ios_base::openmode which = ios_base::out);
    basic_spanstream(const basic_spanstream& rhs) = delete;
    basic_spanstream(basic_spanstream&& rhs) noexcept;

    // \ref{spanstream.assign}, assign and swap:
    basic_spanstream& operator=(const basic_spanstream& rhs) = delete;
    basic_spanstream& operator=(basic_spanstream&& rhs) noexcept;
    void swap(basic_spanstream& rhs) noexcept;

    // \ref{spanstream.members}, members:
    basic_spanbuf<charT, traits>* rdbuf() const noexcept;

    span<charT> span() const noexcept;
	template<ptrdiff_t Extent>
    void span(span<charT> s) noexcept;
  private:
    basic_spanbuf<charT, traits> sb; // \expos
  };

  template <class charT, class traits>
    void swap(basic_spanstream<charT, traits>& x,
              basic_spanstream<charT, traits>& y) noexcept;
}
\end{codeblock}

\pnum
The class
\tcode{basic_spanstream<charT, traits>}
supports reading from and writing to objects of class
\tcode{span<charT, traits>}.
It uses a
\tcode{basic_spanbuf<charT, traits>}
object to control the associated span.
For the sake of exposition, the maintained data is presented here as:
\begin{itemize}
\item
\tcode{sb}, the \tcode{spanbuf} object.
\end{itemize}

%\rSec3[spanstream.cons]{\tcode{basic_spanstream} constructors}
\subsection{27.x.5.1 \tcode{basic_spanstream} constructors [spanstream.cons]}
\label{spanstream.cons}

%\indexlibrary{\idxcode{basic_spanstream}!constructor}%
\begin{itemdecl}
template <ptrdiff_t Extent>
explicit basic_spanstream(
  span<charT, Extent> span,
  ios_base::openmode which = ios_base::out | ios_bas::in);
\end{itemdecl}

\begin{itemdescr}
\pnum
\effects
Constructs an object of class
\tcode{basic_spanstream<charT, traits>},
initializing the base class with
\tcode{basic_iostream(\&sb)}
and initializing \tcode{sb} with
\tcode{basic_spanbuf<charT, traits>{span, which})}~(\ref{spanbuf.cons}).
\end{itemdescr}

%\indexlibrary{\idxcode{basic_spanstream}!constructor}%
\begin{itemdecl}
basic_spanstream(basic_spanstream&& rhs) noexcept;
\end{itemdecl}

\begin{itemdescr}
\pnum
\effects Move constructs from the rvalue \tcode{rhs}. This
is accomplished by move constructing the base class, and the contained
\tcode{basic_spanbuf}.
Next \tcode{basic_istream<charT, traits>::set_rdbuf(\&sb)} is called to
install the contained \tcode{basic_spanbuf}.
\end{itemdescr}

%\rSec3[spanstream.assign]{Assign and swap}
\subsection{27.x.5.2 Assign and swap [spanstream.assign]}
\label{spanstream.assign}

%\indexlibrarymember{operator=}{basic_spanstream}%
\begin{itemdecl}
basic_spanstream& operator=(basic_spanstream&& rhs) noexcept;
\end{itemdecl}

\begin{itemdescr}
\pnum
\effects Move assigns the base and members of \tcode{*this} from the base and corresponding
members of \tcode{rhs}.

\pnum
\returns \tcode{*this}.
\end{itemdescr}

%\indexlibrarymember{swap}{basic_spanstream}%
\begin{itemdecl}
void swap(basic_spanstream& rhs) noexcept;
\end{itemdecl}

\begin{itemdescr}
\pnum
\effects Exchanges the state of \tcode{*this} and
\tcode{rhs} by calling
\tcode{basic_iostream<charT, traits>::swap(rhs)} and
\tcode{sb.swap(rhs.sb)}.
\end{itemdescr}


%\indexlibrarymember{swap}{basic_spanstream}%
\begin{itemdecl}
template <class charT, class traits, class Allocator>
  void swap(basic_spanstream<charT, traits, Allocator>& x,
            basic_spanstream<charT, traits, Allocator>& y) noexcept;
\end{itemdecl}

\begin{itemdescr}
\pnum
\effects As if by \tcode{x.swap(y)}.
\end{itemdescr}

%\rSec3[spanstream.members]{Member functions}
\subsection{27.x.5.3 Member functions [spanstream.members]}
\label{spanstream.members}

%\indexlibrarymember{rdbuf}{basic_spanstream}%
\begin{itemdecl}
basic_spanbuf<charT>* rdbuf() const noexcept;
\end{itemdecl}

\begin{itemdescr}
\pnum
\returns
\tcode{const_cast<basic_spanbuf<charT>*>(\&sb)}.
\end{itemdescr}

%\indexlibrarymember{str}{basic_spanstream}%
\begin{itemdecl}
span<charT> span() const noexcept;
\end{itemdecl}

\begin{itemdescr}
\pnum
\returns
\tcode{rdbuf()->span()}.
\end{itemdescr}

%\indexlibrarymember{str}{basic_spanstream}%
\begin{itemdecl}
template<ptrdiff_t Extent>
void span(span<charT, Extent> s) noexcept;
\end{itemdecl}

\begin{itemdescr}
\pnum
\effects
Calls
\tcode{rdbuf()->span(s)}.
\end{itemdescr}



\chapter{Appendix: Example Implementations}
An example implementation will become available under the author's github account at:
\url{https://github.com/PeterSommerlad/SC22WG21_Papers/tree/master/workspace/Test_basic_spanbuf}
\end{document}

