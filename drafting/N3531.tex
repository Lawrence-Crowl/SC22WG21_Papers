\documentclass[ebook,11pt,article]{memoir}
\usepackage{geometry}                % See geometry.pdf to learn the layout options. There are lots.
\geometry{a4paper}                   % ... or a4paper or a5paper or ... 
%\geometry{landscape}                % Activate for for rotated page geometry
%\usepackage[parfill]{parskip}    % Activate to begin paragraphs with an empty line rather than an indent

\usepackage[final]
           {listings}     % code listings
\usepackage{color}        % define colors for strikeouts and underlines
\usepackage{underscore}   % remove special status of '_' in ordinary text
\usepackage{xspace}
\pagestyle{myheadings}
\markboth{N3531 2013-03-08}{N3531 2013-03-08}
%%TODO
% namespace std { inline namespace literals { inline namespace chrono_literals, string_literals, complex_literals
% put b on the side
% complex: i_f, i, il optional i_l

\title{User-defined Literals for Standard Library Types (version 3)}
\author{Peter Sommerlad}
\date{2013-03-08}                                           % Activate to display a given date or no date
% Definitions and redefinitions of special commands

%%--------------------------------------------------
%% Difference markups
\definecolor{addclr}{rgb}{0,.6,.3} %% 0,.6,.6 was to blue for my taste :-)
\definecolor{remclr}{rgb}{1,0,0}
\definecolor{noteclr}{rgb}{0,0,1}

\renewcommand{\added}[1]{\textcolor{addclr}{\uline{#1}}}
\newcommand{\removed}[1]{\textcolor{remclr}{\sout{#1}}}
\renewcommand{\changed}[2]{\removed{#1}\added{#2}}

\newcommand{\nbc}[1]{[#1]\ }
\newcommand{\addednb}[2]{\added{\nbc{#1}#2}}
\newcommand{\removednb}[2]{\removed{\nbc{#1}#2}}
\newcommand{\changednb}[3]{\removednb{#1}{#2}\added{#3}}
\newcommand{\remitem}[1]{\item\removed{#1}}

\newcommand{\ednote}[1]{\textcolor{noteclr}{[Editor's note: #1] }}
% \newcommand{\ednote}[1]{}

\newenvironment{addedblock}
{
\color{addclr}
}
{
\color{black}
}
\newenvironment{removedblock}
{
\color{remclr}
}
{
\color{black}
}

%%--------------------------------------------------
%% Sectioning macros.  
% Each section has a depth, an automatically generated section 
% number, a name, and a short tag.  The depth is an integer in 
% the range [0,5].  (If it proves necessary, it wouldn't take much
% programming to raise the limit from 5 to something larger.)


% The basic sectioning command.  Example:
%    \Sec1[intro.scope]{Scope}
% defines a first-level section whose name is "Scope" and whose short
% tag is intro.scope.  The square brackets are mandatory.
\def\Sec#1[#2]#3{%
\ifcase#1\let\s=\chapter
      \or\let\s=\section
      \or\let\s=\subsection
      \or\let\s=\subsubsection
      \or\let\s=\paragraph
      \or\let\s=\subparagraph
      \fi%
\s[#3]{#3\hfill[#2]}\label{#2}}

% A convenience feature (mostly for the convenience of the Project
% Editor, to make it easy to move around large blocks of text):
% the \rSec macro is just like the \Sec macro, except that depths 
% relative to a global variable, SectionDepthBase.  So, for example,
% if SectionDepthBase is 1,
%   \rSec1[temp.arg.type]{Template type arguments}
% is equivalent to
%   \Sec2[temp.arg.type]{Template type arguments}
\newcounter{SectionDepthBase}
\newcounter{scratch}

\def\rSec#1[#2]#3{%
\setcounter{scratch}{#1}
\addtocounter{scratch}{\value{SectionDepthBase}}
\Sec{\arabic{scratch}}[#2]{#3}}

\newcommand{\synopsis}[1]{\textbf{#1}}

%%--------------------------------------------------
% Indexing

% locations
\newcommand{\indextext}[1]{\index[generalindex]{#1}}
\newcommand{\indexlibrary}[1]{\index[libraryindex]{#1}}
\newcommand{\indexgram}[1]{\index[grammarindex]{#1}}
\newcommand{\indeximpldef}[1]{\index[impldefindex]{#1}}

\newcommand{\indexdefn}[1]{\indextext{#1}}
\newcommand{\indexgrammar}[1]{\indextext{#1}\indexgram{#1}}
\newcommand{\impldef}[1]{\indeximpldef{#1}implementation-defined}

% appearance
\newcommand{\idxcode}[1]{#1@\tcode{#1}}
\newcommand{\idxhdr}[1]{#1@\tcode{<#1>}}
\newcommand{\idxgram}[1]{#1@\textit{#1}}

%%--------------------------------------------------
% General code style
\newcommand{\CodeStyle}{\ttfamily}
\newcommand{\CodeStylex}[1]{\texttt{#1}}

% Code and definitions embedded in text.
\newcommand{\tcode}[1]{\CodeStylex{#1}}
\newcommand{\techterm}[1]{\textit{#1}\xspace}
\newcommand{\defnx}[2]{\indexdefn{#2}\textit{#1}\xspace}
\newcommand{\defn}[1]{\defnx{#1}{#1}}
\newcommand{\term}[1]{\textit{#1}\xspace}
\newcommand{\grammarterm}[1]{\textit{#1}\xspace}
\newcommand{\placeholder}[1]{\textit{#1}}
\newcommand{\placeholdernc}[1]{\textit{#1\nocorr}}

%%--------------------------------------------------
%% allow line break if needed for justification
\newcommand{\brk}{\discretionary{}{}{}}
%  especially for scope qualifier
\newcommand{\colcol}{\brk::\brk}

%%--------------------------------------------------
%% Macros for funky text
\newcommand{\Cpp}{\texorpdfstring{C\kern-0.05em\protect\raisebox{.35ex}{\textsmaller[2]{+\kern-0.05em+}}}{C++}\xspace}
\newcommand{\CppIII}{\Cpp 2003\xspace}
\newcommand{\CppXI}{\Cpp 2011\xspace}
\newcommand{\CppXIV}{\Cpp 2014\xspace}
\newcommand{\opt}{{\ensuremath{_\mathit{opt}}}\xspace}
\newcommand{\shl}{<{<}}
\newcommand{\shr}{>{>}}
\newcommand{\dcr}{-{-}}
\newcommand{\exor}{\^{}}
\newcommand{\bigoh}[1]{\ensuremath{\mathscr{O}(#1)}}

% Make all tildes a little larger to avoid visual similarity with hyphens.
% FIXME: Remove \tilde in favour of \~.
\renewcommand{\tilde}{\textasciitilde}
\renewcommand{\~}{\textasciitilde}
\let\OldTextAsciiTilde\textasciitilde
\renewcommand{\textasciitilde}{\protect\raisebox{-0.17ex}{\larger\OldTextAsciiTilde}}

%%--------------------------------------------------
%% States and operators
\newcommand{\state}[2]{\tcode{#1}\ensuremath{_{#2}}}
\newcommand{\bitand}{\ensuremath{\, \mathsf{bitand} \,}}
\newcommand{\bitor}{\ensuremath{\, \mathsf{bitor} \,}}
\newcommand{\xor}{\ensuremath{\, \mathsf{xor} \,}}
\newcommand{\rightshift}{\ensuremath{\, \mathsf{rshift} \,}}
\newcommand{\leftshift}[1]{\ensuremath{\, \mathsf{lshift}_#1 \,}}

%% Notes and examples
\newcommand{\EnterBlock}[1]{[\,\textit{#1:}\xspace}
\newcommand{\ExitBlock}[1]{\textit{\,---\,end #1}\,]\xspace}
\newcommand{\enternote}{\EnterBlock{Note}}
\newcommand{\exitnote}{\ExitBlock{note}}
\newcommand{\enterexample}{\EnterBlock{Example}}
\newcommand{\exitexample}{\ExitBlock{example}}
%newer versions, legacy above!
\newcommand{\noteintro}[1]{[\,\textit{#1:}\space}
\newcommand{\noteoutro}[1]{\textit{\,---\,end #1}\,]}
\newenvironment{note}[1][Note]{\noteintro{#1}}{\noteoutro{note}\xspace}
\newenvironment{example}[1][Example]{\noteintro{#1}}{\noteoutro{example}\xspace}

%% Library function descriptions
\newcommand{\Fundescx}[1]{\textit{#1}\xspace}
\newcommand{\Fundesc}[1]{\Fundescx{#1:}}
\newcommand{\required}{\Fundesc{Required behavior}}
\newcommand{\requires}{\Fundesc{Requires}}
\newcommand{\effects}{\Fundesc{Effects}}
\newcommand{\postconditions}{\Fundesc{Postconditions}}
\newcommand{\postcondition}{\Fundesc{Postcondition}}
\newcommand{\preconditions}{\requires}
\newcommand{\precondition}{\requires}
\newcommand{\returns}{\Fundesc{Returns}}
\newcommand{\throws}{\Fundesc{Throws}}
\newcommand{\default}{\Fundesc{Default behavior}}
\newcommand{\complexity}{\Fundesc{Complexity}}
\newcommand{\remark}{\Fundesc{Remark}}
\newcommand{\remarks}{\Fundesc{Remarks}}
\newcommand{\realnote}{\Fundesc{Note}}
\newcommand{\realnotes}{\Fundesc{Notes}}
\newcommand{\errors}{\Fundesc{Error conditions}}
\newcommand{\sync}{\Fundesc{Synchronization}}
\newcommand{\implimits}{\Fundesc{Implementation limits}}
\newcommand{\replaceable}{\Fundesc{Replaceable}}
\newcommand{\returntype}{\Fundesc{Return type}}
\newcommand{\cvalue}{\Fundesc{Value}}
\newcommand{\ctype}{\Fundesc{Type}}
\newcommand{\ctypes}{\Fundesc{Types}}
\newcommand{\dtype}{\Fundesc{Default type}}
\newcommand{\ctemplate}{\Fundesc{Class template}}
\newcommand{\templalias}{\Fundesc{Alias template}}

%% Cross reference
\newcommand{\xref}{\textsc{See also:}\xspace}
\newcommand{\xsee}{\textsc{See:}\xspace}

%% NTBS, etc.
\newcommand{\NTS}[1]{\textsc{#1}\xspace}
\newcommand{\ntbs}{\NTS{ntbs}}
\newcommand{\ntmbs}{\NTS{ntmbs}}
\newcommand{\ntwcs}{\NTS{ntwcs}}
\newcommand{\ntcxvis}{\NTS{ntc16s}}
\newcommand{\ntcxxxiis}{\NTS{ntc32s}}

%% Code annotations
\newcommand{\EXPO}[1]{\textit{#1}}
\newcommand{\expos}{\EXPO{exposition only}}
\newcommand{\impdef}{\EXPO{implementation-defined}}
\newcommand{\impdefnc}{\EXPO{implementation-defined\nocorr}}
\newcommand{\impdefx}[1]{\indeximpldef{#1}\EXPO{implementation-defined}}
\newcommand{\notdef}{\EXPO{not defined}}

\newcommand{\UNSP}[1]{\textit{\texttt{#1}}}
\newcommand{\UNSPnc}[1]{\textit{\texttt{#1}\nocorr}}
\newcommand{\unspec}{\UNSP{unspecified}}
\newcommand{\unspecnc}{\UNSPnc{unspecified}}
\newcommand{\unspecbool}{\UNSP{unspecified-bool-type}}
\newcommand{\seebelow}{\UNSP{see below}}
\newcommand{\seebelownc}{\UNSPnc{see below}}
\newcommand{\unspecuniqtype}{\UNSP{unspecified unique type}}
\newcommand{\unspecalloctype}{\UNSP{unspecified allocator type}}

\newcommand{\EXPLICIT}{\textit{\texttt{EXPLICIT}}}

%% Manual insertion of italic corrections, for aligning in the presence
%% of the above annotations.
\newlength{\itcorrwidth}
\newlength{\itletterwidth}
\newcommand{\itcorr}[1][]{%
 \settowidth{\itcorrwidth}{\textit{x\/}}%
 \settowidth{\itletterwidth}{\textit{x\nocorr}}%
 \addtolength{\itcorrwidth}{-1\itletterwidth}%
 \makebox[#1\itcorrwidth]{}%
}

%% Double underscore
\newcommand{\ungap}{\kern.5pt}
\newcommand{\unun}{\_\ungap\_}
\newcommand{\xname}[1]{\tcode{\unun\ungap#1}}
\newcommand{\mname}[1]{\tcode{\unun\ungap#1\ungap\unun}}

%% Ranges
\newcommand{\Range}[4]{\tcode{#1#3,~\brk{}#4#2}\xspace}
\newcommand{\crange}[2]{\Range{[}{]}{#1}{#2}}
\newcommand{\brange}[2]{\Range{(}{]}{#1}{#2}}
\newcommand{\orange}[2]{\Range{(}{)}{#1}{#2}}
\newcommand{\range}[2]{\Range{[}{)}{#1}{#2}}

%% Change descriptions
\newcommand{\diffdef}[1]{\hfill\break\textbf{#1:}\xspace}
\newcommand{\change}{\diffdef{Change}}
\newcommand{\rationale}{\diffdef{Rationale}}
\newcommand{\effect}{\diffdef{Effect on original feature}}
\newcommand{\difficulty}{\diffdef{Difficulty of converting}}
\newcommand{\howwide}{\diffdef{How widely used}}

%% Miscellaneous
\newcommand{\uniquens}{\textrm{\textit{\textbf{unique}}}}
\newcommand{\stage}[1]{\item{\textbf{Stage #1:}}\xspace}
\newcommand{\doccite}[1]{\textit{#1}\xspace}
\newcommand{\cvqual}[1]{\textit{#1}}
\newcommand{\cv}{\cvqual{cv}}
\renewcommand{\emph}[1]{\textit{#1}\xspace}
\newcommand{\numconst}[1]{\textsl{#1}\xspace}
\newcommand{\logop}[1]{{\footnotesize #1}\xspace}

%%--------------------------------------------------
%% Environments for code listings.

% We use the 'listings' package, with some small customizations.  The
% most interesting customization: all TeX commands are available
% within comments.  Comments are set in italics, keywords and strings
% don't get special treatment.

\lstset{language=C++,
        basicstyle=\small\CodeStyle,
        keywordstyle=,
        stringstyle=,
        xleftmargin=1em,
        showstringspaces=false,
        commentstyle=\itshape\rmfamily,
        columns=flexible,
        keepspaces=true,
        texcl=true}

% Our usual abbreviation for 'listings'.  Comments are in 
% italics.  Arbitrary TeX commands can be used if they're 
% surrounded by @ signs.
\newcommand{\CodeBlockSetup}{
 \lstset{escapechar=@}
 \renewcommand{\tcode}[1]{\textup{\CodeStylex{##1}}}
 \renewcommand{\techterm}[1]{\textit{\CodeStylex{##1}}}
 \renewcommand{\term}[1]{\textit{##1}}
 \renewcommand{\grammarterm}[1]{\textit{##1}}
}

\lstnewenvironment{codeblock}{\CodeBlockSetup}{}

% A code block in which single-quotes are digit separators
% rather than character literals.
\lstnewenvironment{codeblockdigitsep}{
 \CodeBlockSetup
 \lstset{deletestring=[b]{'}}
}{}

% Permit use of '@' inside codeblock blocks (don't ask)
\makeatletter
\newcommand{\atsign}{@}
\makeatother

%%--------------------------------------------------
%% Indented text
\newenvironment{indented}
{\list{}{}\item\relax}
{\endlist}

%%--------------------------------------------------
%% Library item descriptions
\lstnewenvironment{itemdecl}
{
 \lstset{escapechar=@,
 xleftmargin=0em,
 aboveskip=2ex,
 belowskip=0ex	% leave this alone: it keeps these things out of the
				% footnote area
 }
}
{
}

\newenvironment{itemdescr}
{
 \begin{indented}}
{
 \end{indented}
}


%%--------------------------------------------------
%% Bnf environments
\newlength{\BnfIndent}
\setlength{\BnfIndent}{\leftmargini}
\newlength{\BnfInc}
\setlength{\BnfInc}{\BnfIndent}
\newlength{\BnfRest}
\setlength{\BnfRest}{2\BnfIndent}
\newcommand{\BnfNontermshape}{\small\rmfamily\itshape}
\newcommand{\BnfTermshape}{\small\ttfamily\upshape}
\newcommand{\nonterminal}[1]{{\BnfNontermshape #1}}

\newenvironment{bnfbase}
 {
 \newcommand{\nontermdef}[1]{\nonterminal{##1}\indexgrammar{\idxgram{##1}}:}
 \newcommand{\terminal}[1]{{\BnfTermshape ##1}\xspace}
 \newcommand{\descr}[1]{\normalfont{##1}}
 \newcommand{\bnfindentfirst}{\BnfIndent}
 \newcommand{\bnfindentinc}{\BnfInc}
 \newcommand{\bnfindentrest}{\BnfRest}
 \begin{minipage}{.9\hsize}
 \newcommand{\br}{\hfill\\}
 \frenchspacing
 }
 {
 \nonfrenchspacing
 \end{minipage}
 }

\newenvironment{BnfTabBase}[1]
{
 \begin{bnfbase}
 #1
 \begin{indented}
 \begin{tabbing}
 \hspace*{\bnfindentfirst}\=\hspace{\bnfindentinc}\=\hspace{.6in}\=\hspace{.6in}\=\hspace{.6in}\=\hspace{.6in}\=\hspace{.6in}\=\hspace{.6in}\=\hspace{.6in}\=\hspace{.6in}\=\hspace{.6in}\=\hspace{.6in}\=\kill}
{
 \end{tabbing}
 \end{indented}
 \end{bnfbase}
}

\newenvironment{bnfkeywordtab}
{
 \begin{BnfTabBase}{\BnfTermshape}
}
{
 \end{BnfTabBase}
}

\newenvironment{bnftab}
{
 \begin{BnfTabBase}{\BnfNontermshape}
}
{
 \end{BnfTabBase}
}

\newenvironment{simplebnf}
{
 \begin{bnfbase}
 \BnfNontermshape
 \begin{indented}
}
{
 \end{indented}
 \end{bnfbase}
}

\newenvironment{bnf}
{
 \begin{bnfbase}
 \list{}
	{
	\setlength{\leftmargin}{\bnfindentrest}
	\setlength{\listparindent}{-\bnfindentinc}
	\setlength{\itemindent}{\listparindent}
	}
 \BnfNontermshape
 \item\relax
}
{
 \endlist
 \end{bnfbase}
}

% non-copied versions of bnf environments
\newenvironment{ncbnftab}
{
 \begin{bnftab}
}
{
 \end{bnftab}
}

\newenvironment{ncsimplebnf}
{
 \begin{simplebnf}
}
{
 \end{simplebnf}
}

\newenvironment{ncbnf}
{
 \begin{bnf}
}
{
 \end{bnf}
}

%%--------------------------------------------------
%% Drawing environment
%
% usage: \begin{drawing}{UNITLENGTH}{WIDTH}{HEIGHT}{CAPTION}
\newenvironment{drawing}[4]
{
\newcommand{\mycaption}{#4}
\begin{figure}[h]
\setlength{\unitlength}{#1}
\begin{center}
\begin{picture}(#2,#3)\thicklines
}
{
\end{picture}
\end{center}
\caption{\mycaption}
\end{figure}
}

%%--------------------------------------------------
%% Environment for imported graphics
% usage: \begin{importgraphic}{CAPTION}{TAG}{FILE}

\newenvironment{importgraphic}[3]
{%
\newcommand{\cptn}{#1}
\newcommand{\lbl}{#2}
\begin{figure}[htp]\centering%
\includegraphics[scale=.35]{#3}
}
{
\caption{\cptn}\label{\lbl}%
\end{figure}}

%% enumeration display overrides
% enumerate with lowercase letters
\newenvironment{enumeratea}
{
 \renewcommand{\labelenumi}{\alph{enumi})}
 \begin{enumerate}
}
{
 \end{enumerate}
}

% enumerate with arabic numbers
\newenvironment{enumeraten}
{
 \renewcommand{\labelenumi}{\arabic{enumi})}
 \begin{enumerate}
}
{
 \end{enumerate}
}

%%--------------------------------------------------
%% Definitions section
% usage: \definition{name}{xref}
%\newcommand{\definition}[2]{\rSec2[#2]{#1}}
% for ISO format, use:
\newcommand{\definition}[2]{%
\subsection[#1]{\hfill[#2]}\vspace{-.3\onelineskip}\label{#2}\textbf{#1}\\%
}
\newcommand{\definitionx}[2]{%
\subsubsection[#1]{\hfill[#2]}\vspace{-.3\onelineskip}\label{#2}\textbf{#1}\\%
}
\newcommand{\defncontext}[1]{\textlangle#1\textrangle}
 
 %% adopted from standard's layout.tex
 \newcounter{Paras}
\counterwithin{Paras}{chapter}
\counterwithin{Paras}{section}
\counterwithin{Paras}{subsection}
\counterwithin{Paras}{subsubsection}
\counterwithin{Paras}{paragraph}
\counterwithin{Paras}{subparagraph}

 \makeatletter
\def\pnum{\addtocounter{Paras}{1}\noindent\llap{{%
  \scriptsize\raisebox{.7ex}{\arabic{Paras}}}\hspace{\@totalleftmargin}\quad}}
\makeatother

%% PS: add some helpers for coloring, assumes \usepackage{color}
%% should no longer be used, since we have those macros already!
\newcommand{\del}[1]{\removed{#1}}
\newcommand{\ins}[1]{\added{#1}}

\newenvironment{insrt}{\begin{addedblock}}{\end{addedblock}}


\setsecnumdepth{subsection}

\begin{document}
\maketitle
\begin{tabular}[t]{|l|l|}\hline 
Document Number: &  N3531 (update of N3468/N3402)\\\hline
Date: & 2013-03-08 \\\hline
Project: & Programming Language C++\\\hline 
\end{tabular}
\chapter{Changes from N3468}
\begin{itemize}
\item move implementation code to appendix.
\item add discussion on a suffix for \tcode{std::string_ref}.
\item refer to SI units abbreviations definition.
\end{itemize}

\chapter{Changes from N3402}
\begin{itemize}
\item drop binary literals and ask core/evolution first if it would be done by core. it should be done in code via a prefix "\tcode{0b}".
\item drop real-part \tcode{std::complex} literals \tcode{operator"" r()} and make the imaginary-part operators \tcode{i}, \tcode{il}, and \tcode{i_f}.
\item drop mechanics for type deduction of integers from standard. They should be part of type traits anyway and should be also an integral\_constant. This paper still provides their updated implementation as an example for potential implementors.
\item make the floating point representation type of \tcode{chrono::duration} floating point literals unspecified.
\end{itemize}


\chapter{Introduction}
The standard library is lacking pre-defined user-defined literals, even though the standard reserves names not starting with an underscore for it. Even the sequence of papers that introduced UDL to the standard contained useful examples of suffixes for creating values of standard types such as \tcode{s} for \tcode{std::string}, \tcode{h} for \tcode{std::chrono::hours} and \tcode{i} for imaginary parts of complex numbers.

Discussion on the reflector in May 2012 and in Portland Oct 2012 showed demand for some or even many pre-defined UDL operators in the standard library, however, there was no consensus how far to go and how to resolve conflicts in naming. One can summarize the requirements of the discussion as follows:
\begin{itemize}
\item use an inline namespace for a (group of related) UDL operator(s)
\item use an inline namespace \tcode{std::literals} for all such UDL namespaces
\item ISO units would be nice to have, but some might conflict with existing syntax, such as \tcode{F}, \tcode{l}, \tcode{lm}, \tcode{lx}, \tcode{''}(seconds) or cannot be represented easily in all fonts, such as $\Omega$ or $^{\circ}\mathrm{C}$.\footnote{see at {http://www.ewh.ieee.org/soc/ias/pub-dept/abbreviation.pdf}}
\item suffix \tcode{s} was proposed for \tcode{std::string} but is also ISO standard for seconds and could be convenient for \tcode{std::chrono::duration} values. However, because the overloads for such an \tcode{operator"" s()} differ for these two usage scenarios, \tcode{s} can also be used for \tcode{std::string}.
\item an UDL for constructing \tcode{std::string} literals should not allocate memory, but use a \tcode{string_ref} type, once some like that is available in the standard. Such a facility could easily provide a different suffix to make string literals a \tcode{string_ref} instead of a \tcode{std::basic_string}. 
\item any proposal that is made for adding user-defined literal functions to the standard library will evoke some discussion.
\item Alberto Ganesh Barbati <albertobarbati@gmail.com> suggested to provide the number parsing facility to be used by UDL template operators should be exported, so that authors of UDL suffixes could reuse it. Discussion in Portland showed it should be left to the implementer and the general usability of such a feature is limited to the (few) experts implementing UDL template operators. Boost might be a place to provide such a reusable expert-only feature. In addition the \tcode{select_int_type} should be an \tcode{std::integral_constant} and should be part of the type traits, but under a better name.
\item BSI suggested to change suffixes for complex literals to the style of \tcode{i_f}, however, the current suffixes where heavily discussed in Portland and the result of the library group decision is used for this paper.
\item BSI suggested to use suffix s for \tcode{std::string_ref} once it is accepted to the standard library. However, I believe we want keep \tcode{s} for meaning \tcode{std::string} and \tcode{std::string_ref} should propose a different suffix such as \tcode{sr} for denoting short-hand for converting string literals into type \tcode{std::string_ref}. I suggest the \tcode{string_ref} proposal adds its own suffix mimicking the approach of this proposal.
\end{itemize}

Based on this discussion this paper proposes to include UDL operators for the following library components.
\begin{itemize}
\item \tcode{std::basic_string}, suffix \tcode{s} in inline namespace \tcode{std::literals::string_literals}
\item \tcode{std::complex}, suffixes \tcode{i, il, i_f} in inline namespace \tcode{std::literals::complex_literals}
\item \tcode{std::chrono::duration}, suffixes \tcode{h, min, s, ms, us, ns} in \\inline namespace \tcode{std::literals::chrono_literals}
\end{itemize}

\section{Rationale}
User-defined literal operators (UDL) are a new features of C++11. However, while the feature is there it is not yet used by the standard library of C++11. The papers introducing UDL already named a few examples where source code could benefit from pre-defined UDL operators in the library, such as imaginary number, or std::string literals.

Fortunately the C++11 standard already reserved UDL names not starting with an underscore '_' for future standardization. 

Several library classes representing scalar or numeric types can benefit from pre-defined UDL operators that ease their use: \tcode{std::complex} and \tcode{std::chrono::duration}. Also \tcode{std::basic_string<CharT>} instantiations are a viable candidate for a suffix \tcode{operator "" s(CharT const*, size_t)}.

\section{Open Issues Discussed}
\subsection{Suffixes Utilities - exposition only}
It has to be decided if the utilities for implementing UDL suffix operators with integers should be standardized.
Discussion in Portland consented in abandoning that for this paper. It also gave consensus to put the binary literals in the hand of the compiler with a prefix of \tcode{0b}, e.g., \tcode{0b1001} instead.

The template \tcode{select_int_type} might be a candidate for the clause [meta.type.synop], aka header \tcode{<type_traits>} but with another name.

\subsection{Upper-case versions of suffixes - dropped}
%While it seems useful and symmetric to provide upper case variations of suffixes \tcode{u, l, ll, ull} as allowed for integral constants, it needs to be discussed if also \tcode{'b'} should vary in case accordingly and thus doubling the number of overloaded UDL operators.

The original version said: "To avoid combinatorial explosion there won't be upper case version of the UDL suffixes unless they are mimicking built-in suffixes." Since we moved binary to the compiler, this no longer applies for more than one suffix. So we only propose lower case suffixes only, even for \tcode{std::complex}.

%Similar discussions might be needed for complex numbers suffixes.

%I have the opinion we should stick for lower case only for strings and chrono suffixes. 

\subsection{Suffix r for real-part only std::complex numbers - dropped}
It needs to be discussed if this set of suffixes (r, lr, fr, R, LR, FR) for complex numbers with a real part only is actually required and useful. If all viable overloaded versions of constexpr operators are available for std::complex they might not be needed.

Discussion in Portland consented to abandon the \tcode{r} suffixes, because they are redundant, once some more operators of \tcode{std::complex} will be made \tcode{constexpr}.

%\section{}
\section{Acknowledgements}
Acknowledgements go to the original authors of the sequence of papers the lead to inclusion of UDL in the standard and to the participants of the discussion on UDL on the reflector. Special thanks to Daniel Kr\"ugler for tremendous feedback on all drafts and to Jonathan Wakely for guidelines on GCC command line options. Thanks to Alberto Ganesh Barbati for feedback on duration representation overflow and suggestion for also providing the number parsing as a standardized library component. Thanks to Bjarne Stroustrup for suggesting to add more rationale to the proposal.

Thanks to all participants in the discussions in groups "library" and "evolution" in Portland.

Thanks to the BSI reviewers of N3468 and Roger Orr as their spokesperson.


\chapter{Proposed Library Additions}
It must be decided in which section to actually put the proposed changes. I suggest we add them to the corresponding library parts, where appropriate.
\section{namespace literals for collecting standard UDLs}
As a common schema this paper proposes to put all suffixes for user defined literals in separate inline namespaces that are below the inline namespace \tcode{std::literals}. 
\enternote
This allows a user either to do a \tcode{using namespace std::literals;} to import all literal operators from the standard available through header file inclusion, or to use \tcode{using namespace std::string_literals;} to just obtain the literals operators for a specific type.
\exitnote


%%%%
%%%%%

\section{operator"" s() for basic_string}
Make the following additions and changes to library clause 21 [strings] to accommodate the user-defined literal suffix s for string literals resulting in a corresponding string object instead of array of characters.

Insert in 21.3 [string.classes] in the synopsis at the appropriate place the inline namespace \tcode{std::literals::string_literals}
\begin{codeblock}
namespace std{
inline namespace literals{
inline namespace string_literals{
basic_string<char> operator "" s(char const *str, size_t len);
basic_string<wchar_t> operator "" s(wchar_t const *str, size_t len);
basic_string<char16_t> operator "" s(char16_t const *str, size_t len);
basic_string<char32_t> operator "" s(char32_t const *str, size_t len);
}}}
\end{codeblock}

Before subclause 21.7 [c.strings] add a new subclause [basic.string.literals]

\rSec1[basic.string.literals]{Suffix for basic_string literals}
\begin{itemdecl}
basic_string<char> operator "" s(char const *str, size_t len);
\end{itemdecl}

\begin{itemdescr}
\pnum
\returns
\tcode{basic_string<char>\{str,len\}}
\end{itemdescr}

\begin{itemdecl}
basic_string<wchar_t> operator "" s(wchar_t const *str, size_t len);
\end{itemdecl}
\begin{itemdescr}
\pnum
\returns
\tcode{basic_string<wchar_t>\{str,len\}}
\end{itemdescr}

\begin{itemdecl}
basic_string<char16_t> operator "" s(char16_t const *str, size_t len);
\end{itemdecl}
\begin{itemdescr}
\pnum
\returns
\tcode{basic_string<char16_t>\{str,len\}}
\end{itemdescr}

\begin{itemdecl}
basic_string<char32_t> operator "" s(char32_t const *str, size_t len);
\end{itemdecl}
\begin{itemdescr}
\pnum
\returns
\tcode{basic_string<char32_t>\{str,len\}}
\end{itemdescr}

\section{Imaginary Literal Suffixes for std::complex}
Make the following additions and changes to library subclause 26.4 [complex.numbers] to accommodate user-defined literal suffixes for complex number literals.

Insert in subclause 26.4.1 [complex.syn] in the synopsis at the appropriate place the namespace std::literals::complex_literals
\begin{codeblock}
namespace std{
inline namespace literals{
inline namespace complex_literals{
constexpr complex<long double> operator"" il(long double);
constexpr complex<long double> operator"" il(unsigned long long);
constexpr complex<double> operator"" i(long double);
constexpr complex<double> operator"" i(unsigned long long);
constexpr complex<float> operator"" i_f(long double);
constexpr complex<float> operator"" i_f(unsigned long long);
}}}
\end{codeblock}

Insert a new subclause before subclause 26.4.9 [ccmplx] as follows
\rSec1[complex.literals]{Suffix for complex number literals}
\pnum
This section describes literal suffixes for constructing complex number literals. The suffixes i, il, i_f create complex numbers with their imaginary part denoted by the given literal number and the real part being zero of the types \tcode{complex<double>}, \tcode{complex<long double>}, and \tcode{complex<float>} respectively. 

\begin{itemdecl}
constexpr complex<long double> operator"" il(long double d);
constexpr complex<long double> operator"" il(unsigned long long d);
\end{itemdecl}

\begin{itemdescr}
\pnum
\effects
Creates a complex literal as \tcode{complex<long double>\{0.0L, static_cast<long double>(d)\}}.
\end{itemdescr}

\begin{itemdecl}
constexpr complex<double> operator"" i(long double d);
constexpr complex<double> operator"" i(unsigned long long d);
\end{itemdecl}

\begin{itemdescr}
\pnum
\effects
Creates a complex literal as \tcode{complex<double>\{0.0, static_cast<double>(d)\}}.
\end{itemdescr}

\begin{itemdecl}
constexpr complex<float> operator"" i_f(long double d);
constexpr complex<float> operator"" i_f(unsigned long long d);
\end{itemdecl}

\begin{itemdescr}
\pnum
\effects
Creates a complex literal as \tcode{complex<float>\{0.0f, static_cast<float>(d)\}}.
\enternote
The keyword \tcode{if} is not available as a suffix.
\exitnote
\end{itemdescr}


\section{Suffixes for chrono::duration values}
Make the following additions and changes to library subclause 20.11 [time] to accommodate user-defined literal suffixes for chrono::duration literals.

Insert in subclause 20.11.2 [time.syn] in the synopsis at the appropriate place the inline namespace \tcode{std::literals::chrono_literals}.
\renewcommand{\unspec}{\UNSP{unspecified}}%%without \xspace for nicer code layout

\begin{codeblock}
namespace std {
inline namespace literals {
inline namespace chrono_literals{
constexpr 
chrono::hours operator"" h(unsigned long long);
constexpr 
chrono::duration<@\unspec@, ratio<3600,1>> operator"" h(long double);
constexpr 
chrono::minutes operator"" min(unsigned long long);
constexpr 
chrono::duration<@\unspec@, ratio<60,1>> operator"" min(long double);
constexpr 
chrono::seconds operator"" s(unsigned long long);
constexpr 
chrono::duration<@\unspec@> operator"" s(long double);
constexpr 
chrono::milliseconds operator"" ms(unsigned long long);
constexpr 
chrono::duration<@\unspec@, milli> operator"" ms(long double);
constexpr 
chrono::microseconds operator"" us(unsigned long long);
constexpr 
chrono::duration<@\unspec@, micro> operator"" us(long double);
constexpr 
chrono::nanoseconds operator"" ns(unsigned long long);
constexpr 
chrono::duration<@\unspec@, nano> operator"" ns(long double);
}}}
\end{codeblock}

Insert in subclause 20.11.5 [time.duration] after subclause 20.11.5.7 [time.duration.cast] a new subclause 20.11.5.8 [time.duration.literals] as follows.

\rSec2[time.duration.literals]{Suffix for duration literals}
\pnum
This section describes literal suffixes for constructing duration literals. The suffixes \tcode{h,min,s,ms,us,ns} denote duration values of the corresponding types \tcode{hours}, \tcode{minutes}, \tcode{seconds}, \tcode{miliseconds}, \tcode{microseconds}, and \tcode{nanoseconds} respectively if they are applied to integral literals. 

\pnum
If any of these suffixes are applied to a floating point literal the result is a \tcode{chrono::duration} literal with an unspecified floating point representation.

\pnum
If any of these suffixes are applied to an integer literal and the resulting \tcode{chrono::duration} value cannot be represented in the result type because of overflow, the program is ill-formed.

\pnum
\enterexample 
The following code shows some duration literals.
\begin{codeblock}
{
    using namespace std::chrono_literals;
    auto constexpr aday=24h; 
    auto constexpr lesson=45min; 
    auto constexpr halfanhour=0.5h;
}
\end{codeblock}
\exitexample

\pnum
\enternote
The suffix for microseconds is \tcode{us}, but if unicode identifiers are allowed, implementations are encouraged to provide $\mu{}$\tcode{s} as well.
\exitnote

\begin{itemdecl}
constexpr 
chrono::hours operator"" h(unsigned long long hours);
constexpr 
chrono::duration<@\unspec@, ratio<3600,1>> operator"" h(long double hours);
\end{itemdecl}

\begin{itemdescr}
\pnum
\effects
Creates a \tcode{duration} literal representing \tcode{hours} hours. 
\end{itemdescr}

\begin{itemdecl}
constexpr 
chrono::minutes operator"" min(unsigned long long min);
constexpr 
chrono::duration<@\unspec@, ratio<60,1>> operator"" min(long double min);
\end{itemdecl}

\begin{itemdescr}
\pnum
\effects
Creates a \tcode{duration} literal representing \tcode{min} minutes. 
\end{itemdescr}

\begin{itemdecl}
constexpr 
chrono::seconds operator"" s(unsigned long long sec);
constexpr 
chrono::duration<@\unspec@> operator"" s(long double sec);
\end{itemdecl}

\begin{itemdescr}
\pnum
\effects
Creates a \tcode{duration} literal representing \tcode{sec} seconds. 

\enternote
The same suffix \tcode{s} is used for \tcode{basic_string} but there is no conflict, since duration suffixes always apply to numbers and string literal suffixes always apply to character array literals.
\exitnote
\end{itemdescr}

\begin{itemdecl}
constexpr 
chrono::milliseconds operator"" ms(unsigned long long msec);
constexpr 
chrono::duration<@\unspec@, milli> operator"" ms(long double msec);
\end{itemdecl}

\begin{itemdescr}
\pnum
\effects
Creates a \tcode{duration} literal representing \tcode{msec} milliseconds. 
\end{itemdescr}

\begin{itemdecl}
constexpr 
chrono::microseconds operator"" us(unsigned long long usec);
constexpr 
chrono::duration<@\unspec@, micro> operator"" us(long double usec);
\end{itemdecl}

\begin{itemdescr}
\pnum
\effects
Creates a \tcode{duration} literal representing \tcode{usec} microseconds. 
\end{itemdescr}

\begin{itemdecl}
constexpr 
chrono::nanoseconds operator"" ns(unsigned long long nsec);
constexpr 
chrono::duration<@\unspec@, nano> operator"" ns(long double nsec);
\end{itemdecl}

\begin{itemdescr}
\pnum
\effects
Creates a \tcode{duration} literal representing \tcode{nsec} nanoseconds. 
\end{itemdescr}

\newpage
\chapter{Appendix: Possible Implementation}
This section shows some possible implementations of the user-defined-literals proposed.
\section{Compile-time Integer Parsing}
For its usage, see the implementation of \tcode{std::chrono::duration} literals. This part will not be standardized but it is for exposition of the technique. It is planned to provide this facility through a future Boost library with adapted namespace names.
\begin{codeblock}
#ifndef SUFFIXESPARSENUMBERS_H_
#define SUFFIXESPARSENUMBERS_H_
#include <cstddef>
namespace std {
namespace parse_int {
template <unsigned base, char... Digits>
struct parse_int{
    static_assert(base<=16u,"only support up to hexadecimal");
    static_assert(! sizeof...(Digits), "invalid integral constant");
    static constexpr unsigned long long value=0;
};

template <char... Digits>
struct base_dispatch;

template <char... Digits>
struct base_dispatch<'0','x',Digits...>{
    static constexpr unsigned long long value=parse_int<16u,Digits...>::value;
};
template <char... Digits>
struct base_dispatch<'0','X',Digits...>{
    static constexpr unsigned long long value=parse_int<16u,Digits...>::value;
};
template <char... Digits>
struct base_dispatch<'0',Digits...>{
    static constexpr unsigned long long value=parse_int<8u,Digits...>::value;
};
template <char... Digits>
struct base_dispatch{
    static constexpr unsigned long long value=parse_int<10u,Digits...>::value;
};

constexpr unsigned long long
pow(unsigned base, size_t to) {
    return to?(to%2?base:1)*pow(base,to/2)*pow(base,to/2):1;
}

template <unsigned base, char... Digits>
struct parse_int<base,'0',Digits...>{
    static constexpr unsigned long long value{ parse_int<base,Digits...>::value};
};
template <unsigned base, char... Digits>
struct parse_int<base,'1',Digits...>{
    static constexpr unsigned long long value{ 1 *pow(base,sizeof...(Digits))
                                               + parse_int<base,Digits...>::value};
};
template <unsigned base, char... Digits>
struct parse_int<base,'2',Digits...>{
    static_assert(base>2,"invalid digit");
    static constexpr unsigned long long value{ 2 *pow(base,sizeof...(Digits))
                                               + parse_int<base,Digits...>::value};
};
template <unsigned base, char... Digits>
struct parse_int<base,'3',Digits...>{
    static_assert(base>3,"invalid digit");
    static constexpr unsigned long long value{ 3 *pow(base,sizeof...(Digits))
                                               + parse_int<base,Digits...>::value};
};
template <unsigned base, char... Digits>
struct parse_int<base,'4',Digits...>{
    static_assert(base>4,"invalid digit");
    static constexpr unsigned long long value{ 4 *pow(base,sizeof...(Digits))
                                               + parse_int<base,Digits...>::value};
};
template <unsigned base, char... Digits>
struct parse_int<base,'5',Digits...>{
    static_assert(base>5,"invalid digit");
    static constexpr unsigned long long value{ 5 *pow(base,sizeof...(Digits))
                                               + parse_int<base,Digits...>::value};
};
template <unsigned base, char... Digits>
struct parse_int<base,'6',Digits...>{
    static_assert(base>6,"invalid digit");
    static constexpr unsigned long long value{ 6 *pow(base,sizeof...(Digits))
                                               + parse_int<base,Digits...>::value};
};
template <unsigned base, char... Digits>
struct parse_int<base,'7',Digits...>{
    static_assert(base>7,"invalid digit");
    static constexpr unsigned long long value{ 7 *pow(base,sizeof...(Digits))
                                               + parse_int<base,Digits...>::value};
};
template <unsigned base, char... Digits>
struct parse_int<base,'8',Digits...>{
    static_assert(base>8,"invalid digit");
    static constexpr unsigned long long value{ 8 *pow(base,sizeof...(Digits))
                                               + parse_int<base,Digits...>::value};
};
template <unsigned base, char... Digits>
struct parse_int<base,'9',Digits...>{
    static_assert(base>9,"invalid digit");
    static constexpr unsigned long long value{ 9 *pow(base,sizeof...(Digits))
                                               + parse_int<base,Digits...>::value};
};
template <unsigned base, char... Digits>
struct parse_int<base,'a',Digits...>{
    static_assert(base>0xa,"invalid digit");

    static constexpr unsigned long long value{ 0xa *pow(base,sizeof...(Digits))
                                               + parse_int<base,Digits...>::value};
};
template <unsigned base, char... Digits>
struct parse_int<base,'b',Digits...>{
    static_assert(base>0xb,"invalid digit");
    static constexpr unsigned long long value{ 0xb *pow(base,sizeof...(Digits))
                                               + parse_int<base,Digits...>::value};
};
template <unsigned base, char... Digits>
struct parse_int<base,'c',Digits...>{
    static_assert(base>0xc,"invalid digit");
    static constexpr unsigned long long value{ 0xc *pow(base,sizeof...(Digits))
                                               + parse_int<base,Digits...>::value};
};
template <unsigned base, char... Digits>
struct parse_int<base,'d',Digits...>{
    static_assert(base>0xd,"invalid digit");
    static constexpr unsigned long long value{ 0xd *pow(base,sizeof...(Digits))
                                               + parse_int<base,Digits...>::value};
};
template <unsigned base, char... Digits>
struct parse_int<base,'e',Digits...>{
    static_assert(base>0xe,"invalid digit");
    static constexpr unsigned long long value{ 0xe *pow(base,sizeof...(Digits))
                                               + parse_int<base,Digits...>::value};
};
template <unsigned base, char... Digits>
struct parse_int<base,'f',Digits...>{
    static_assert(base>0xf,"invalid digit");
    static constexpr unsigned long long value{ 0xf *pow(base,sizeof...(Digits))
                                               + parse_int<base,Digits...>::value};
};
template <unsigned base, char... Digits>
struct parse_int<base,'A',Digits...>{
    static_assert(base>0xA,"invalid digit");
    static constexpr unsigned long long value{ 0xa *pow(base,sizeof...(Digits))
                                               + parse_int<base,Digits...>::value};
};
template <unsigned base, char... Digits>
struct parse_int<base,'B',Digits...>{
    static_assert(base>0xB,"invalid digit");
    static constexpr unsigned long long value{ 0xb *pow(base,sizeof...(Digits))
                                               + parse_int<base,Digits...>::value};
};
template <unsigned base, char... Digits>
struct parse_int<base,'C',Digits...>{
    static_assert(base>0xC,"invalid digit");
    static constexpr unsigned long long value{ 0xc *pow(base,sizeof...(Digits))
                                               + parse_int<base,Digits...>::value};
};
template <unsigned base, char... Digits>
struct parse_int<base,'D',Digits...>{
    static_assert(base>0xD,"invalid digit");
    static constexpr unsigned long long value{ 0xd *pow(base,sizeof...(Digits))
                                               + parse_int<base,Digits...>::value};
};
template <unsigned base, char... Digits>
struct parse_int<base,'E',Digits...>{
    static_assert(base>0xE,"invalid digit");
    static constexpr unsigned long long value{ 0xe *pow(base,sizeof...(Digits))
                                               + parse_int<base,Digits...>::value};
};
template <unsigned base, char... Digits>
struct parse_int<base,'F',Digits...>{
    static_assert(base>0xF,"invalid digit");
    static constexpr unsigned long long value{ 0xf *pow(base,sizeof...(Digits))
                                               + parse_int<base,Digits...>::value};
};
}}
#endif /* SUFFIXESPARSENUMBERS_H_ */
\end{codeblock}

Here comes some example code from my test cases showing the use of that facility:
\begin{codeblock}
template <char... Digits>
constexpr  unsigned long long
operator"" _ternary(){
    return std::parse_int::parse_int<3,Digits...>::value;
}
constexpr auto five= 012_ternary;
static_assert(five==5, "_ternary should be three-based");
//constexpr auto invalid=3_ternary;

template <char... Digits>
constexpr  unsigned long long
operator"" _testit(){
    return std::suffixes::base_dispatch<Digits...>::value;
}

constexpr auto a = 123_testit;   // value 123
constexpr auto b = 0123_testit;  // value 0123
constexpr auto c = 0x123_testit; // value 0x123

\end{codeblock}


\section{integral type fitting}
During discussion in Portland it was assumed that this facility should be part of the \tcode{type_traits} header, because it provides similar facilities. The discussion further provided a simplified implementation of it through inheriting from \tcode{std::integral_constant}. For the purpose of this paper it is exposition only for the technique of it.

It is planned to provide it via Boost for others to try and use.

\begin{codeblock}
#ifndef SELECT_INT_TYPE_H_
#define SELECT_INT_TYPE_H_
#include <type_traits>
#include <limits>

namespace std {
namespace select_int_type {

template <unsigned long long val, typename... INTS>
struct select_int_type;

template <unsigned long long val, typename INTTYPE, typename... INTS>
struct select_int_type<val,INTTYPE,INTS...>
:integral_constant<typename conditional<
(val<=static_cast<unsigned long long>(std::numeric_limits<INTTYPE>::max()))
,INTTYPE
,typename select_int_type<val,INTS...>::value_type >::type , val> {
};

template <unsigned long long val>
struct select_int_type<val>:integral_constant<unsigned long long,val>{
};

}}
#endif /* SELECT_INT_TYPE_H_ */
\end{codeblock}

Here are some examples from my test cases to show an example on how to use. For others see the following section on binary literals which we will not standardize as is.

\begin{codeblock}
using std::select_int_type::select_int_type;
template <unsigned long long val>
constexpr
typename select_int_type<val,
short, int, long long>::value_type
foo() {
    return  select_int_type<val,
            short, int, long long>::value;
}
static_assert(std::is_same<decltype(foo<100>()), 
              short>::value,"foo<100>() is short");
static_assert(std::is_same<decltype(foo<0x10000>()), int>::value,
              "foo<0x10000>() is int");
static_assert(std::is_same<decltype(foo<0x100000000000>()), long long>::value,
              "foo<0x100000000000>() is long long");
\end{codeblock}

\section{binary}
This is exposition only and no longer considered to be standardized. A version of it might be provided through Boost as _b and corresponding versions. It is used to demonstrate the possible facilities, but it was considered in Portland that it could easily add to compile times, if binary literals would require to use that facility. Users who can not wait for binary literals to be implemented by their compiler can use it for their needs until then.
\begin{codeblock}
#ifndef BINARY_H_
#define BINARY_H_
#include <limits>
#include <type_traits>
#include "select_int_type.h"
namespace std{
namespace suffixes{
namespace binary{
namespace __impl{

template <char... Digits>
struct bitsImpl{
	static_assert(! sizeof...(Digits),
			"binary literal digits must be 0 or 1");
	static constexpr unsigned long long value=0;
};

template <char... Digits>
struct bitsImpl<'0',Digits...>{
	static constexpr unsigned long long value=bitsImpl<Digits...>::value;
};

template <char... Digits>
struct bitsImpl<'1',Digits...>{
	static constexpr unsigned long long value=
			bitsImpl<Digits...>::value|(1ULL<<sizeof...(Digits));
};
using std::select_int_type::select_int_type;
}

template <char... Digits>
constexpr typename
__impl::select_int_type<__impl::bitsImpl<Digits...>::value,
      int, unsigned, long, unsigned long, long long>::value_type
operator"" b(){
	return	__impl::select_int_type<__impl::bitsImpl<Digits...>::value,
			int, unsigned, long, unsigned long, long long>::value;
}
template <char... Digits>
constexpr typename
__impl::select_int_type<__impl::bitsImpl<Digits...>::value,
      long, unsigned long, long long>::value_type
operator"" bl(){
	return	__impl::select_int_type<__impl::bitsImpl<Digits...>::value,
			      long, unsigned long, long long>::value;
}
template <char... Digits>
constexpr auto
operator"" bL() -> decltype(operator "" bl<Digits...>()){
	return 	operator "" bl<Digits...>();
}

template <char... Digits>
constexpr typename
__impl::select_int_type<__impl::bitsImpl<Digits...>::value,
       long long>::value_type
operator"" bll(){
	return 	__impl::select_int_type<__impl::bitsImpl<Digits...>::value,
			      long long>::value;
}
template <char... Digits>
constexpr auto
operator"" bLL() -> decltype(operator "" bll<Digits...>()){
	return 	operator "" bll<Digits...>();
}

template <char... Digits>
constexpr typename
__impl::select_int_type<__impl::bitsImpl<Digits...>::value,
      unsigned, unsigned long>::value_type
operator"" bu(){
	return 	__impl::select_int_type<__impl::bitsImpl<Digits...>::value,
			      unsigned, unsigned long>::value;
}

template <char... Digits>
constexpr auto
operator"" bU() -> decltype(operator "" bu<Digits...>()){
	return 	operator "" bu<Digits...>();
}

template <char... Digits>
constexpr typename
__impl::select_int_type<__impl::bitsImpl<Digits...>::value,
       unsigned long>::value_type
operator"" bul(){
	return 	__impl::select_int_type<__impl::bitsImpl<Digits...>::value,
			      unsigned long>::value;
}
template <char... Digits>
constexpr auto
operator"" bUL() -> decltype(operator "" bul<Digits...>()){
	return 	operator "" bul<Digits...>();
}
template <char... Digits>
constexpr auto
operator"" buL() -> decltype(operator "" bul<Digits...>()){
	return 	operator "" bul<Digits...>();
}
template <char... Digits>
constexpr auto
operator"" bUl() -> decltype(operator "" bul<Digits...>()){
	return 	operator "" bul<Digits...>();
}
template <char... Digits>
constexpr unsigned long long
operator"" bull(){
	return __impl::bitsImpl<Digits...>::value;
}
template <char... Digits>
constexpr unsigned long long
operator"" bULL(){
	return __impl::bitsImpl<Digits...>::value;
}
template <char... Digits>
constexpr unsigned long long
operator"" buLL(){
	return __impl::bitsImpl<Digits...>::value;
}
template <char... Digits>
constexpr unsigned long long
operator"" bUll(){
	return __impl::bitsImpl<Digits...>::value;
}
} // binary
} //suffixes
} // std

#endif /* BINARY_H_ */
\end{codeblock}

%%%%%%%%%%

\section{basic_string}
This section is for exposition only to show two possible implementations. The macro version can be briefer, the non-macro version is more elaborate and duplicates code, otherwise they are equivalent. 

\begin{codeblock}
#ifndef STRING_SUFFIX_H_
#define STRING_SUFFIX_H_
#include <string>
namespace std{
inline namespace literals{
inline namespace string_literals{
#if 0
#define __MAKE_SUFFIX_S(CHAR) \
	basic_string<CHAR>\
operator "" s(CHAR const *str, size_t len){\
	return basic_string<CHAR>(str,len);\
}

__MAKE_SUFFIX_S(char)
__MAKE_SUFFIX_S(wchar_t)
__MAKE_SUFFIX_S(char16_t)
__MAKE_SUFFIX_S(char32_t)
#undef __MAKE_SUFFIX
#else // copy-paste version for proposal

basic_string<char>
operator "" s(char const *str, size_t len){
return basic_string<char>(str,len);
}
basic_string<wchar_t>
operator "" s(wchar_t const *str, size_t len){
return basic_string<wchar_t>(str,len);
}
basic_string<char16_t>
operator "" s(char16_t const *str, size_t len){
return basic_string<char16_t>(str,len);
}
basic_string<char32_t>
operator "" s(char32_t const *str, size_t len){
return basic_string<char32_t>(str,len);
}
#endif
}}}
#endif /* STRING_SUFFIX_H_ */
\end{codeblock}

Here are some test cases testing compilability of the UDL operators for \tcode{std::basic_string}.

\begin{codeblock}
using namespace std::string_literals;
static_assert(std::is_same<decltype("hallo"s),std::string>{},
              "s means std::string");
static_assert(std::is_same<decltype(u8"hallo"s),std::string>{},
              "u8 s means std::string");
static_assert(std::is_same<decltype(L"hallo"s),std::wstring>{},
              "L s means std::wstring");
static_assert(std::is_same<decltype(u"hallo"s),std::u16string>{},
              "u s means std::u16string");
static_assert(std::is_same<decltype(U"hallo"s),std::u32string>{},
              "U s means std::u32string");

void testStringSuffix(){
	ASSERT_EQUAL(typeid("hi"s).name(),typeid(std::string).name());
	ASSERT_EQUAL(std::string{"hello"},"hello"s);
}
\end{codeblock}

%%%%%%%%%%%%%%%%%
\section{std::complex}
This is the example implementation of UDL operators for creating imaginary values of type \tcode{std::complex}. Note that this deviates from the original proposal which put the letter indicating the type in front of the i. This caused the problem that you couldn't create a \tcode{std::complex<float>} from a hexadecimal integer, as \tcode{0x1fi} would have been interpreted as \tcode{std::complex<double>{0,15}} instead of \tcode{std::complex<float>{0,1}}. And because "if" is a keyword, we decided to use "i_f" instead.
\begin{codeblock}
#ifndef COMPLEX_SUFFIX_H_
#define COMPLEX_SUFFIX_H_
#include <complex>
namespace std{
inline namespace literals{
inline namespace complex_literals{
constexpr
std::complex<long double> operator"" il(long double d){
	return std::complex<long double>{0.0L,static_cast<long double>(d)};
}
constexpr
std::complex<long double> operator"" il(unsigned long long d){
	return std::complex<long double>{0.0L,static_cast<long double>(d)};
}
constexpr
std::complex<double> operator"" i(long double d){
	return std::complex<double>{0,static_cast<double>(d)};
}
constexpr
std::complex<double> operator"" i(unsigned long long d){
	return std::complex<double>{0,static_cast<double>(d)};
}
constexpr
std::complex<float> operator"" i_f(long double d){
	return std::complex<float>{0,static_cast<float>(d)};
}
constexpr
std::complex<float> operator"" i_f(unsigned long long d){
	return std::complex<float>{0,static_cast<float>(d)};
}
}}}
#endif /* COMPLEX_SUFFIX_H_ */
\end{codeblock}

\section{duration}
Except for inline namespaces this hasn't been changed from the original version in N3402. And it includes the fix of the missing static member definition of "value".
\begin{codeblock}
#ifndef CHRONO_SUFFIX_H_
#define CHRONO_SUFFIX_H_
#include <chrono>
#include <limits>
#include "suffixes_parse_integers.h"
namespace std {
inline namespace literals {
inline namespace chrono_literals{

namespace __impl {
using namespace std::parse_int;

template <unsigned long long val, typename DUR>
struct select_type:conditional<
    (val <=static_cast<unsigned long long>(
            std::numeric_limits<typename DUR::rep>::max()))
	, DUR , void > {
		static constexpr typename select_type::type
			value{ static_cast<typename select_type::type>(val) };
};
template <unsigned long long val, typename DUR>
constexpr typename select_type<val,DUR>::type select_type<val,DUR>::value;
}

template <char... Digits>
constexpr typename
__impl::select_type<__impl::base_dispatch<Digits...>::value,
std::chrono::hours>::type
operator"" h(){
	return  __impl::select_type<__impl::base_dispatch<Digits...>::value,
	        std::chrono::hours>::value;
}
constexpr std::chrono::duration<long double, ratio<3600,1>>
operator"" h(long double hours){
	return std::chrono::duration<long double,ratio<3600,1>>{hours};
}
template <char... Digits>
constexpr typename
__impl::select_type<__impl::base_dispatch<Digits...>::value,
std::chrono::minutes>::type
operator"" min(){
	return __impl::select_type<__impl::base_dispatch<Digits...>::value,
	       std::chrono::minutes>::value;
}
constexpr std::chrono::duration<long double, ratio<60,1>>
operator"" min(long double min){
	return std::chrono::duration<long double,ratio<60,1>>{min};
}

template <char... Digits>
constexpr typename
__impl::select_type<__impl::base_dispatch<Digits...>::value,
std::chrono::seconds>::type
operator"" s(){
	return __impl::select_type<__impl::base_dispatch<Digits...>::value,
	       std::chrono::seconds>::value;
}
constexpr std::chrono::duration<long double>
operator"" s(long double sec){
	return std::chrono::duration<long double>{sec};
}

template <char... Digits>
constexpr typename
__impl::select_type<__impl::base_dispatch<Digits...>::value,
std::chrono::milliseconds>::type
operator"" ms(){
	return __impl::select_type<__impl::base_dispatch<Digits...>::value,
	       std::chrono::milliseconds>::value;
}
constexpr std::chrono::duration<long double, milli>
operator"" ms(long double msec){
	return std::chrono::duration<long double,milli>{msec};
}

template <char... Digits>
constexpr typename
__impl::select_type<__impl::base_dispatch<Digits...>::value,
std::chrono::microseconds>::type
operator"" us(){
	return __impl::select_type<__impl::base_dispatch<Digits...>::value,
	       std::chrono::microseconds>::value;
}
constexpr std::chrono::duration<long double, micro>
operator"" us(long double usec){
	return std::chrono::duration<long double, micro>{usec};
}

template <char... Digits>
constexpr typename
__impl::select_type<__impl::base_dispatch<Digits...>::value,
std::chrono::nanoseconds>::type
operator"" ns(){
	return __impl::select_type<__impl::base_dispatch<Digits...>::value,
	       std::chrono::nanoseconds>::value;
}
constexpr std::chrono::duration<long double, nano>
operator"" ns(long double nsec){
	return std::chrono::duration<long double, nano>{nsec};
}

}}}
#endif /* CHRONO_SUFFIX_H_ */
\end{codeblock}

Here are some test cases for the UDL operators to show how they can be applied/tested. Most just test compilability.
\begin{codeblock}

using namespace std::chrono_literals;

void testChronoLiterals(){
	static_assert(std::is_same<std::chrono::hours::rep, int>::value,
	              "hours are too long to check");
	//constexpr auto overflowh= 0x80000000h; // compile error!
	constexpr auto xh=5h;
	ASSERT_EQUAL(std::chrono::hours{5}.count(),xh.count());
	static_assert(std::chrono::hours{5}==xh,"chrono suffix hours");
	constexpr auto xmin=0x5min;
	static_assert(std::chrono::duration<unsigned long long,
	              std::ratio<60,1>>{5}==xmin,"chrono suffix min");
	constexpr auto x=05s;
	static_assert(std::chrono::duration<unsigned long long,
	              std::ratio<1,1>>{5}==x,"chrono suffix s");
	constexpr auto xms=5ms;
	static_assert(std::chrono::duration<unsigned long long,
	              std::ratio<1,1000>>{5}==xms,"chrono suffix ms");
	constexpr auto xus=5us;
	static_assert(std::chrono::duration<unsigned long long,
	              std::ratio<1,1000000>>{5}==xus,"chrono suffix ms");
	constexpr auto xns=5ns;
	static_assert(std::chrono::duration<unsigned long long,
	              std::ratio<1,1000000000>>{5}==xns,"chrono suffix ms");

	constexpr auto dh=0.5h;
	constexpr auto dmin=0.5min;
	constexpr auto ds=0.5s;
	constexpr auto dms=0.5ms;
	constexpr auto dus=0.5us;
	constexpr auto dns=0.5ns;
}
void aTestForDuration(){
	auto  x=5h;
	auto y=18000s;
	ASSERT_EQUAL(x,y);
}
\end{codeblock}

\end{document}   