\documentclass[ebook,11pt,article]{memoir}
\usepackage{geometry}                % See geometry.pdf to learn the layout options. There are lots.
\geometry{a4paper}                   % ... or a4paper or a5paper or ... 
%\geometry{landscape}                % Activate for for rotated page geometry
%\usepackage[parfill]{parskip}    % Activate to begin paragraphs with an empty line rather than an indent

\usepackage[final]
           {listings}     % code listings
\usepackage{color}        % define colors for strikeouts and underlines
\usepackage{underscore}   % remove special status of '_' in ordinary text
\usepackage{xspace}



\title{User-defined Literals for Standard Library Types}
\author{Peter Sommerlad}
\date{2012-08-31}                                           % Activate to display a given date or no date
% Definitions and redefinitions of special commands

%%--------------------------------------------------
%% Difference markups
\definecolor{addclr}{rgb}{0,.6,.3} %% 0,.6,.6 was to blue for my taste :-)
\definecolor{remclr}{rgb}{1,0,0}
\definecolor{noteclr}{rgb}{0,0,1}

\renewcommand{\added}[1]{\textcolor{addclr}{\uline{#1}}}
\newcommand{\removed}[1]{\textcolor{remclr}{\sout{#1}}}
\renewcommand{\changed}[2]{\removed{#1}\added{#2}}

\newcommand{\nbc}[1]{[#1]\ }
\newcommand{\addednb}[2]{\added{\nbc{#1}#2}}
\newcommand{\removednb}[2]{\removed{\nbc{#1}#2}}
\newcommand{\changednb}[3]{\removednb{#1}{#2}\added{#3}}
\newcommand{\remitem}[1]{\item\removed{#1}}

\newcommand{\ednote}[1]{\textcolor{noteclr}{[Editor's note: #1] }}
% \newcommand{\ednote}[1]{}

\newenvironment{addedblock}
{
\color{addclr}
}
{
\color{black}
}
\newenvironment{removedblock}
{
\color{remclr}
}
{
\color{black}
}

%%--------------------------------------------------
%% Sectioning macros.  
% Each section has a depth, an automatically generated section 
% number, a name, and a short tag.  The depth is an integer in 
% the range [0,5].  (If it proves necessary, it wouldn't take much
% programming to raise the limit from 5 to something larger.)


% The basic sectioning command.  Example:
%    \Sec1[intro.scope]{Scope}
% defines a first-level section whose name is "Scope" and whose short
% tag is intro.scope.  The square brackets are mandatory.
\def\Sec#1[#2]#3{%
\ifcase#1\let\s=\chapter
      \or\let\s=\section
      \or\let\s=\subsection
      \or\let\s=\subsubsection
      \or\let\s=\paragraph
      \or\let\s=\subparagraph
      \fi%
\s[#3]{#3\hfill[#2]}\label{#2}}

% A convenience feature (mostly for the convenience of the Project
% Editor, to make it easy to move around large blocks of text):
% the \rSec macro is just like the \Sec macro, except that depths 
% relative to a global variable, SectionDepthBase.  So, for example,
% if SectionDepthBase is 1,
%   \rSec1[temp.arg.type]{Template type arguments}
% is equivalent to
%   \Sec2[temp.arg.type]{Template type arguments}
\newcounter{SectionDepthBase}
\newcounter{scratch}

\def\rSec#1[#2]#3{%
\setcounter{scratch}{#1}
\addtocounter{scratch}{\value{SectionDepthBase}}
\Sec{\arabic{scratch}}[#2]{#3}}

\newcommand{\synopsis}[1]{\textbf{#1}}

%%--------------------------------------------------
% Indexing

% locations
\newcommand{\indextext}[1]{\index[generalindex]{#1}}
\newcommand{\indexlibrary}[1]{\index[libraryindex]{#1}}
\newcommand{\indexgram}[1]{\index[grammarindex]{#1}}
\newcommand{\indeximpldef}[1]{\index[impldefindex]{#1}}

\newcommand{\indexdefn}[1]{\indextext{#1}}
\newcommand{\indexgrammar}[1]{\indextext{#1}\indexgram{#1}}
\newcommand{\impldef}[1]{\indeximpldef{#1}implementation-defined}

% appearance
\newcommand{\idxcode}[1]{#1@\tcode{#1}}
\newcommand{\idxhdr}[1]{#1@\tcode{<#1>}}
\newcommand{\idxgram}[1]{#1@\textit{#1}}

%%--------------------------------------------------
% General code style
\newcommand{\CodeStyle}{\ttfamily}
\newcommand{\CodeStylex}[1]{\texttt{#1}}

% Code and definitions embedded in text.
\newcommand{\tcode}[1]{\CodeStylex{#1}}
\newcommand{\techterm}[1]{\textit{#1}\xspace}
\newcommand{\defnx}[2]{\indexdefn{#2}\textit{#1}\xspace}
\newcommand{\defn}[1]{\defnx{#1}{#1}}
\newcommand{\term}[1]{\textit{#1}\xspace}
\newcommand{\grammarterm}[1]{\textit{#1}\xspace}
\newcommand{\placeholder}[1]{\textit{#1}}
\newcommand{\placeholdernc}[1]{\textit{#1\nocorr}}

%%--------------------------------------------------
%% allow line break if needed for justification
\newcommand{\brk}{\discretionary{}{}{}}
%  especially for scope qualifier
\newcommand{\colcol}{\brk::\brk}

%%--------------------------------------------------
%% Macros for funky text
\newcommand{\Cpp}{\texorpdfstring{C\kern-0.05em\protect\raisebox{.35ex}{\textsmaller[2]{+\kern-0.05em+}}}{C++}\xspace}
\newcommand{\CppIII}{\Cpp 2003\xspace}
\newcommand{\CppXI}{\Cpp 2011\xspace}
\newcommand{\CppXIV}{\Cpp 2014\xspace}
\newcommand{\opt}{{\ensuremath{_\mathit{opt}}}\xspace}
\newcommand{\shl}{<{<}}
\newcommand{\shr}{>{>}}
\newcommand{\dcr}{-{-}}
\newcommand{\exor}{\^{}}
\newcommand{\bigoh}[1]{\ensuremath{\mathscr{O}(#1)}}

% Make all tildes a little larger to avoid visual similarity with hyphens.
% FIXME: Remove \tilde in favour of \~.
\renewcommand{\tilde}{\textasciitilde}
\renewcommand{\~}{\textasciitilde}
\let\OldTextAsciiTilde\textasciitilde
\renewcommand{\textasciitilde}{\protect\raisebox{-0.17ex}{\larger\OldTextAsciiTilde}}

%%--------------------------------------------------
%% States and operators
\newcommand{\state}[2]{\tcode{#1}\ensuremath{_{#2}}}
\newcommand{\bitand}{\ensuremath{\, \mathsf{bitand} \,}}
\newcommand{\bitor}{\ensuremath{\, \mathsf{bitor} \,}}
\newcommand{\xor}{\ensuremath{\, \mathsf{xor} \,}}
\newcommand{\rightshift}{\ensuremath{\, \mathsf{rshift} \,}}
\newcommand{\leftshift}[1]{\ensuremath{\, \mathsf{lshift}_#1 \,}}

%% Notes and examples
\newcommand{\EnterBlock}[1]{[\,\textit{#1:}\xspace}
\newcommand{\ExitBlock}[1]{\textit{\,---\,end #1}\,]\xspace}
\newcommand{\enternote}{\EnterBlock{Note}}
\newcommand{\exitnote}{\ExitBlock{note}}
\newcommand{\enterexample}{\EnterBlock{Example}}
\newcommand{\exitexample}{\ExitBlock{example}}
%newer versions, legacy above!
\newcommand{\noteintro}[1]{[\,\textit{#1:}\space}
\newcommand{\noteoutro}[1]{\textit{\,---\,end #1}\,]}
\newenvironment{note}[1][Note]{\noteintro{#1}}{\noteoutro{note}\xspace}
\newenvironment{example}[1][Example]{\noteintro{#1}}{\noteoutro{example}\xspace}

%% Library function descriptions
\newcommand{\Fundescx}[1]{\textit{#1}\xspace}
\newcommand{\Fundesc}[1]{\Fundescx{#1:}}
\newcommand{\required}{\Fundesc{Required behavior}}
\newcommand{\requires}{\Fundesc{Requires}}
\newcommand{\effects}{\Fundesc{Effects}}
\newcommand{\postconditions}{\Fundesc{Postconditions}}
\newcommand{\postcondition}{\Fundesc{Postcondition}}
\newcommand{\preconditions}{\requires}
\newcommand{\precondition}{\requires}
\newcommand{\returns}{\Fundesc{Returns}}
\newcommand{\throws}{\Fundesc{Throws}}
\newcommand{\default}{\Fundesc{Default behavior}}
\newcommand{\complexity}{\Fundesc{Complexity}}
\newcommand{\remark}{\Fundesc{Remark}}
\newcommand{\remarks}{\Fundesc{Remarks}}
\newcommand{\realnote}{\Fundesc{Note}}
\newcommand{\realnotes}{\Fundesc{Notes}}
\newcommand{\errors}{\Fundesc{Error conditions}}
\newcommand{\sync}{\Fundesc{Synchronization}}
\newcommand{\implimits}{\Fundesc{Implementation limits}}
\newcommand{\replaceable}{\Fundesc{Replaceable}}
\newcommand{\returntype}{\Fundesc{Return type}}
\newcommand{\cvalue}{\Fundesc{Value}}
\newcommand{\ctype}{\Fundesc{Type}}
\newcommand{\ctypes}{\Fundesc{Types}}
\newcommand{\dtype}{\Fundesc{Default type}}
\newcommand{\ctemplate}{\Fundesc{Class template}}
\newcommand{\templalias}{\Fundesc{Alias template}}

%% Cross reference
\newcommand{\xref}{\textsc{See also:}\xspace}
\newcommand{\xsee}{\textsc{See:}\xspace}

%% NTBS, etc.
\newcommand{\NTS}[1]{\textsc{#1}\xspace}
\newcommand{\ntbs}{\NTS{ntbs}}
\newcommand{\ntmbs}{\NTS{ntmbs}}
\newcommand{\ntwcs}{\NTS{ntwcs}}
\newcommand{\ntcxvis}{\NTS{ntc16s}}
\newcommand{\ntcxxxiis}{\NTS{ntc32s}}

%% Code annotations
\newcommand{\EXPO}[1]{\textit{#1}}
\newcommand{\expos}{\EXPO{exposition only}}
\newcommand{\impdef}{\EXPO{implementation-defined}}
\newcommand{\impdefnc}{\EXPO{implementation-defined\nocorr}}
\newcommand{\impdefx}[1]{\indeximpldef{#1}\EXPO{implementation-defined}}
\newcommand{\notdef}{\EXPO{not defined}}

\newcommand{\UNSP}[1]{\textit{\texttt{#1}}}
\newcommand{\UNSPnc}[1]{\textit{\texttt{#1}\nocorr}}
\newcommand{\unspec}{\UNSP{unspecified}}
\newcommand{\unspecnc}{\UNSPnc{unspecified}}
\newcommand{\unspecbool}{\UNSP{unspecified-bool-type}}
\newcommand{\seebelow}{\UNSP{see below}}
\newcommand{\seebelownc}{\UNSPnc{see below}}
\newcommand{\unspecuniqtype}{\UNSP{unspecified unique type}}
\newcommand{\unspecalloctype}{\UNSP{unspecified allocator type}}

\newcommand{\EXPLICIT}{\textit{\texttt{EXPLICIT}}}

%% Manual insertion of italic corrections, for aligning in the presence
%% of the above annotations.
\newlength{\itcorrwidth}
\newlength{\itletterwidth}
\newcommand{\itcorr}[1][]{%
 \settowidth{\itcorrwidth}{\textit{x\/}}%
 \settowidth{\itletterwidth}{\textit{x\nocorr}}%
 \addtolength{\itcorrwidth}{-1\itletterwidth}%
 \makebox[#1\itcorrwidth]{}%
}

%% Double underscore
\newcommand{\ungap}{\kern.5pt}
\newcommand{\unun}{\_\ungap\_}
\newcommand{\xname}[1]{\tcode{\unun\ungap#1}}
\newcommand{\mname}[1]{\tcode{\unun\ungap#1\ungap\unun}}

%% Ranges
\newcommand{\Range}[4]{\tcode{#1#3,~\brk{}#4#2}\xspace}
\newcommand{\crange}[2]{\Range{[}{]}{#1}{#2}}
\newcommand{\brange}[2]{\Range{(}{]}{#1}{#2}}
\newcommand{\orange}[2]{\Range{(}{)}{#1}{#2}}
\newcommand{\range}[2]{\Range{[}{)}{#1}{#2}}

%% Change descriptions
\newcommand{\diffdef}[1]{\hfill\break\textbf{#1:}\xspace}
\newcommand{\change}{\diffdef{Change}}
\newcommand{\rationale}{\diffdef{Rationale}}
\newcommand{\effect}{\diffdef{Effect on original feature}}
\newcommand{\difficulty}{\diffdef{Difficulty of converting}}
\newcommand{\howwide}{\diffdef{How widely used}}

%% Miscellaneous
\newcommand{\uniquens}{\textrm{\textit{\textbf{unique}}}}
\newcommand{\stage}[1]{\item{\textbf{Stage #1:}}\xspace}
\newcommand{\doccite}[1]{\textit{#1}\xspace}
\newcommand{\cvqual}[1]{\textit{#1}}
\newcommand{\cv}{\cvqual{cv}}
\renewcommand{\emph}[1]{\textit{#1}\xspace}
\newcommand{\numconst}[1]{\textsl{#1}\xspace}
\newcommand{\logop}[1]{{\footnotesize #1}\xspace}

%%--------------------------------------------------
%% Environments for code listings.

% We use the 'listings' package, with some small customizations.  The
% most interesting customization: all TeX commands are available
% within comments.  Comments are set in italics, keywords and strings
% don't get special treatment.

\lstset{language=C++,
        basicstyle=\small\CodeStyle,
        keywordstyle=,
        stringstyle=,
        xleftmargin=1em,
        showstringspaces=false,
        commentstyle=\itshape\rmfamily,
        columns=flexible,
        keepspaces=true,
        texcl=true}

% Our usual abbreviation for 'listings'.  Comments are in 
% italics.  Arbitrary TeX commands can be used if they're 
% surrounded by @ signs.
\newcommand{\CodeBlockSetup}{
 \lstset{escapechar=@}
 \renewcommand{\tcode}[1]{\textup{\CodeStylex{##1}}}
 \renewcommand{\techterm}[1]{\textit{\CodeStylex{##1}}}
 \renewcommand{\term}[1]{\textit{##1}}
 \renewcommand{\grammarterm}[1]{\textit{##1}}
}

\lstnewenvironment{codeblock}{\CodeBlockSetup}{}

% A code block in which single-quotes are digit separators
% rather than character literals.
\lstnewenvironment{codeblockdigitsep}{
 \CodeBlockSetup
 \lstset{deletestring=[b]{'}}
}{}

% Permit use of '@' inside codeblock blocks (don't ask)
\makeatletter
\newcommand{\atsign}{@}
\makeatother

%%--------------------------------------------------
%% Indented text
\newenvironment{indented}
{\list{}{}\item\relax}
{\endlist}

%%--------------------------------------------------
%% Library item descriptions
\lstnewenvironment{itemdecl}
{
 \lstset{escapechar=@,
 xleftmargin=0em,
 aboveskip=2ex,
 belowskip=0ex	% leave this alone: it keeps these things out of the
				% footnote area
 }
}
{
}

\newenvironment{itemdescr}
{
 \begin{indented}}
{
 \end{indented}
}


%%--------------------------------------------------
%% Bnf environments
\newlength{\BnfIndent}
\setlength{\BnfIndent}{\leftmargini}
\newlength{\BnfInc}
\setlength{\BnfInc}{\BnfIndent}
\newlength{\BnfRest}
\setlength{\BnfRest}{2\BnfIndent}
\newcommand{\BnfNontermshape}{\small\rmfamily\itshape}
\newcommand{\BnfTermshape}{\small\ttfamily\upshape}
\newcommand{\nonterminal}[1]{{\BnfNontermshape #1}}

\newenvironment{bnfbase}
 {
 \newcommand{\nontermdef}[1]{\nonterminal{##1}\indexgrammar{\idxgram{##1}}:}
 \newcommand{\terminal}[1]{{\BnfTermshape ##1}\xspace}
 \newcommand{\descr}[1]{\normalfont{##1}}
 \newcommand{\bnfindentfirst}{\BnfIndent}
 \newcommand{\bnfindentinc}{\BnfInc}
 \newcommand{\bnfindentrest}{\BnfRest}
 \begin{minipage}{.9\hsize}
 \newcommand{\br}{\hfill\\}
 \frenchspacing
 }
 {
 \nonfrenchspacing
 \end{minipage}
 }

\newenvironment{BnfTabBase}[1]
{
 \begin{bnfbase}
 #1
 \begin{indented}
 \begin{tabbing}
 \hspace*{\bnfindentfirst}\=\hspace{\bnfindentinc}\=\hspace{.6in}\=\hspace{.6in}\=\hspace{.6in}\=\hspace{.6in}\=\hspace{.6in}\=\hspace{.6in}\=\hspace{.6in}\=\hspace{.6in}\=\hspace{.6in}\=\hspace{.6in}\=\kill}
{
 \end{tabbing}
 \end{indented}
 \end{bnfbase}
}

\newenvironment{bnfkeywordtab}
{
 \begin{BnfTabBase}{\BnfTermshape}
}
{
 \end{BnfTabBase}
}

\newenvironment{bnftab}
{
 \begin{BnfTabBase}{\BnfNontermshape}
}
{
 \end{BnfTabBase}
}

\newenvironment{simplebnf}
{
 \begin{bnfbase}
 \BnfNontermshape
 \begin{indented}
}
{
 \end{indented}
 \end{bnfbase}
}

\newenvironment{bnf}
{
 \begin{bnfbase}
 \list{}
	{
	\setlength{\leftmargin}{\bnfindentrest}
	\setlength{\listparindent}{-\bnfindentinc}
	\setlength{\itemindent}{\listparindent}
	}
 \BnfNontermshape
 \item\relax
}
{
 \endlist
 \end{bnfbase}
}

% non-copied versions of bnf environments
\newenvironment{ncbnftab}
{
 \begin{bnftab}
}
{
 \end{bnftab}
}

\newenvironment{ncsimplebnf}
{
 \begin{simplebnf}
}
{
 \end{simplebnf}
}

\newenvironment{ncbnf}
{
 \begin{bnf}
}
{
 \end{bnf}
}

%%--------------------------------------------------
%% Drawing environment
%
% usage: \begin{drawing}{UNITLENGTH}{WIDTH}{HEIGHT}{CAPTION}
\newenvironment{drawing}[4]
{
\newcommand{\mycaption}{#4}
\begin{figure}[h]
\setlength{\unitlength}{#1}
\begin{center}
\begin{picture}(#2,#3)\thicklines
}
{
\end{picture}
\end{center}
\caption{\mycaption}
\end{figure}
}

%%--------------------------------------------------
%% Environment for imported graphics
% usage: \begin{importgraphic}{CAPTION}{TAG}{FILE}

\newenvironment{importgraphic}[3]
{%
\newcommand{\cptn}{#1}
\newcommand{\lbl}{#2}
\begin{figure}[htp]\centering%
\includegraphics[scale=.35]{#3}
}
{
\caption{\cptn}\label{\lbl}%
\end{figure}}

%% enumeration display overrides
% enumerate with lowercase letters
\newenvironment{enumeratea}
{
 \renewcommand{\labelenumi}{\alph{enumi})}
 \begin{enumerate}
}
{
 \end{enumerate}
}

% enumerate with arabic numbers
\newenvironment{enumeraten}
{
 \renewcommand{\labelenumi}{\arabic{enumi})}
 \begin{enumerate}
}
{
 \end{enumerate}
}

%%--------------------------------------------------
%% Definitions section
% usage: \definition{name}{xref}
%\newcommand{\definition}[2]{\rSec2[#2]{#1}}
% for ISO format, use:
\newcommand{\definition}[2]{%
\subsection[#1]{\hfill[#2]}\vspace{-.3\onelineskip}\label{#2}\textbf{#1}\\%
}
\newcommand{\definitionx}[2]{%
\subsubsection[#1]{\hfill[#2]}\vspace{-.3\onelineskip}\label{#2}\textbf{#1}\\%
}
\newcommand{\defncontext}[1]{\textlangle#1\textrangle}
 
 %% adopted from standard's layout.tex
 \newcounter{Paras}
\counterwithin{Paras}{chapter}
\counterwithin{Paras}{section}
\counterwithin{Paras}{subsection}
\counterwithin{Paras}{subsubsection}
\counterwithin{Paras}{paragraph}
\counterwithin{Paras}{subparagraph}

 \makeatletter
\def\pnum{\addtocounter{Paras}{1}\noindent\llap{{%
  \scriptsize\raisebox{.7ex}{\arabic{Paras}}}\hspace{\@totalleftmargin}\quad}}
\makeatother

%% PS: add some helpers for coloring, assumes \usepackage{color}
%% should no longer be used, since we have those macros already!
\newcommand{\del}[1]{\removed{#1}}
\newcommand{\ins}[1]{\added{#1}}

\newenvironment{insrt}{\begin{addedblock}}{\end{addedblock}}


\setsecnumdepth{subsection}

\begin{document}
\maketitle
\begin{tabular}[t]{|l|l|}\hline 
Document Number: &  N3402=12-0092 \\\hline
Date: & 2012-08-31 \\\hline
Project: & Programming Language C++\\\hline 
\end{tabular}

\chapter{Introduction}
The standard library is lacking pre-defined user-defined literals, even though the standard reserves names not starting with an underscore for it. Even the sequence of papers that introduced UDL to the standard contained useful examples of suffixes for creating values of standard types such as s for std::string, b for binary representation of integers and i for imaginary parts of complex numbers.

Discussion on the reflector in May 2012 showed demand for some or even many pre-defined UDL operators in the standard library, however, there was no consensus how far to go and how to resolve conflicts in naming. One can summarize the requirements of the discussion as follows:
\begin{itemize}
\item use a namespace for a (group of related) UDL operator(s)
\item use a namespace within \tcode{std} for all such UDL namespaces, \tcode{std::suffixes} was suggested
\item ISO units would be nice to have, but some might conflict with existing syntax, such as F, l, lm, lx, ''(seconds) or cannot be represented easily in all fonts, such as $\Omega$ or $^{\circ}\mathrm{C}$.
\item \tcode{s} was proposed for \tcode{std::string} but is also ISO standard for seconds and could be convenient for \tcode{std::chrono::duration} values.
\item an UDL for constructing \tcode{std::string} literals should not allocate memory, but use a \tcode{str_ref} type, once some like that is available in the standard.
\item any proposal that is made for adding user-defined literal functions to the standard library will evoke some discussion.
\item Alberto Ganesh Barbati <albertobarbati@gmail.com> suggested to provide the number parsing facility to be used by UDL template operators should be exported, so that authors of UDL suffixes could reuse it.
\end{itemize}

Based on this discussion this paper proposes to include UDL operators for the following library components.
\begin{itemize}
\item \tcode{unsigned} integers, suffix \tcode{b} plus further suffixes denoting resulting types as for integral constants in namespace \tcode{std::suffixes::binary}
\item \tcode{std::basic_string}, suffix \tcode{s} in namespace \tcode{std::suffixes::string}
\item \tcode{std::complex}, suffixes \tcode{i, li, fi, r, lr, fr} in namespace \tcode{std::suffixes::complex}
\item \tcode{std::chrono::duration}, suffixes \tcode{h, min, s, ms, us, ns} in \\namespace \tcode{std::suffixes::chrono}
\end{itemize}

\section{Rationale}
User-defined literal operators (UDL) are a new features of C++11. However, while the feature is there it is not yet used by the standard library of C++11. The papers introducing UDL already named a few examples where source code could benefit from pre-defined UDL operators in the library, such as binary number, imaginary number, or std::string literals.

Fortunately the C++11 standard already reserved UDL names not starting with an underscore '_' for future standardization. 

In addition to a facility for binary literals for integral values, several library classes representing scalar or numeric types can benefit from pre-defined UDL operators that ease their use: \tcode{std::complex} and \tcode{std::chrono::duration}. Also std::basic_string<CharT> instantiations are a viable candidate for a suffix \tcode{operator "" s(CharT const*, size_t)}.

During the creation of this paper, it became apparent that a mechanism for parsing integral values from UDL operator template can be useful in its own, so that implementers of their own UDL operators can reuse it. A further observation from implementing binary literals was, that to mimic the mechanism of the compiler to automatically select a best fitting integral type, based on the literal's value can be reused as well in similar UDL operators. The last mechanism actually requires parsing integral values by UDL operator templates to retain the value as a compile-time constant. This would get lost if the UDL operator taking an \tcode{unsigned long long} would have been chosen.

\section{Open Issues}
\subsection{Suffixes Utilities}
It has to be decided if the utilities for implementing UDL suffix operators with integers should be standardized.

Especially the \tcode{constexpr unsigned long long pow(unsigned, size_t)} function might be put somewhere else, because it is actually unrelated and might not be standardized at all. I started out with the classic recursive template class with a static const value member, before I recognized that a compile-time \tcode{pow()} for small integer bases is actually what is needed.

The template \tcode{select_int_type} might be a candidate for the clause [meta.type.synop], aka header \tcode{<type_traits>}.

\subsection{Upper-case versions of suffixes}
While it seems useful and symmetric to provide upper case variations of suffixes \tcode{u, l, ll, ull} as allowed for integral constants, it needs to be discussed if also \tcode{'b'} should vary in case accordingly and thus doubling the number of overloaded UDL operators.

Similar discussions might be needed for complex numbers suffixes.

I have the opinion we should stick for lower case only for strings and chrono suffixes. 

\subsection{Suffix r for real-part only std::complex numbers}
It needs to be discussed if this set of suffixes (r, lr, fr, R, LR, FR) for complex numbers with a real part only is actually required and useful. If all viable overloaded versions of constexpr operators are available for std::complex they might not be needed.

%\section{}
\section{Acknowledgements}
Acknowledgements go to the original authors of the sequence of papers the lead to inclusion of UDL in the standard and to the participants of the discussion on UDL on the reflector. Special thanks to Daniel Kr\"ugler for feedback on early drafts and to Jonathan Wakely for guidelines on GCC command line options. Thanks to Alberto Ganesh Barbati for feedback on duration representation overflow and suggestion for also providing the number parsing as a standardized library component. Thanks to Bjarne Stroustrup for suggesting to add more rationale to the proposal.

\chapter{Possible Implementation}
This section shows some possible implementations of the user-defined-literals proposed.
\section{integer parsing}
For its usage, seee the implementation of \tcode{std::chrono::duration} literals.
\begin{codeblock}
#ifndef SUFFIXESPARSENUMBERS_H_
#define SUFFIXESPARSENUMBERS_H_
#include <cstddef>
namespace std {
namespace suffixes {
namespace parse_int {

template <unsigned base, char... Digits>
struct parse_int{
    static_assert(base<=16u,"only support up to hexadecimal");
    static_assert(! sizeof...(Digits), "invalid integral constant");
    static constexpr unsigned long long value=0;
};

template <char... Digits>
struct base_dispatch;

template <char... Digits>
struct base_dispatch<'0','x',Digits...>{
    static constexpr unsigned long long value=parse_int<16u,Digits...>::value;
};
template <char... Digits>
struct base_dispatch<'0','X',Digits...>{
    static constexpr unsigned long long value=parse_int<16u,Digits...>::value;
};
template <char... Digits>
struct base_dispatch<'0',Digits...>{
    static constexpr unsigned long long value=parse_int<8u,Digits...>::value;
};
template <char... Digits>
struct base_dispatch{
    static constexpr unsigned long long value=parse_int<10u,Digits...>::value;
};

constexpr unsigned long long
pow(unsigned base, size_t to) {
    return to?base * pow(base,to-1):1;
}

template <unsigned base, char... Digits>
struct parse_int<base,'0',Digits...>{
    static constexpr unsigned long long value{ parse_int<base,Digits...>::value};
};
template <unsigned base, char... Digits>
struct parse_int<base,'1',Digits...>{
    static constexpr unsigned long long value{ 1 *pow(base,sizeof...(Digits))
                                               + parse_int<base,Digits...>::value};
};
template <unsigned base, char... Digits>
struct parse_int<base,'2',Digits...>{
    static_assert(base>2,"invalid digit");
    static constexpr unsigned long long value{ 2 *pow(base,sizeof...(Digits))
                                               + parse_int<base,Digits...>::value};
};
template <unsigned base, char... Digits>
struct parse_int<base,'3',Digits...>{
    static_assert(base>3,"invalid digit");
    static constexpr unsigned long long value{ 3 *pow(base,sizeof...(Digits))
                                               + parse_int<base,Digits...>::value};
};
template <unsigned base, char... Digits>
struct parse_int<base,'4',Digits...>{
    static_assert(base>4,"invalid digit");
    static constexpr unsigned long long value{ 4 *pow(base,sizeof...(Digits))
                                               + parse_int<base,Digits...>::value};
};
template <unsigned base, char... Digits>
struct parse_int<base,'5',Digits...>{
    static_assert(base>5,"invalid digit");
    static constexpr unsigned long long value{ 5 *pow(base,sizeof...(Digits))
                                               + parse_int<base,Digits...>::value};
};
template <unsigned base, char... Digits>
struct parse_int<base,'6',Digits...>{
    static_assert(base>6,"invalid digit");
    static constexpr unsigned long long value{ 6 *pow(base,sizeof...(Digits))
                                               + parse_int<base,Digits...>::value};
};
template <unsigned base, char... Digits>
struct parse_int<base,'7',Digits...>{
    static_assert(base>7,"invalid digit");
    static constexpr unsigned long long value{ 7 *pow(base,sizeof...(Digits))
                                               + parse_int<base,Digits...>::value};
};
template <unsigned base, char... Digits>
struct parse_int<base,'8',Digits...>{
    static_assert(base>8,"invalid digit");
    static constexpr unsigned long long value{ 8 *pow(base,sizeof...(Digits))
                                               + parse_int<base,Digits...>::value};
};
template <unsigned base, char... Digits>
struct parse_int<base,'9',Digits...>{
    static_assert(base>9,"invalid digit");
    static constexpr unsigned long long value{ 9 *pow(base,sizeof...(Digits))
                                               + parse_int<base,Digits...>::value};
};
template <unsigned base, char... Digits>
struct parse_int<base,'a',Digits...>{
    static_assert(base>0xa,"invalid digit");

    static constexpr unsigned long long value{ 0xa *pow(base,sizeof...(Digits))
                                               + parse_int<base,Digits...>::value};
};
template <unsigned base, char... Digits>
struct parse_int<base,'b',Digits...>{
    static_assert(base>0xb,"invalid digit");
    static constexpr unsigned long long value{ 0xb *pow(base,sizeof...(Digits))
                                               + parse_int<base,Digits...>::value};
};
template <unsigned base, char... Digits>
struct parse_int<base,'c',Digits...>{
    static_assert(base>0xc,"invalid digit");
    static constexpr unsigned long long value{ 0xc *pow(base,sizeof...(Digits))
                                               + parse_int<base,Digits...>::value};
};
template <unsigned base, char... Digits>
struct parse_int<base,'d',Digits...>{
    static_assert(base>0xd,"invalid digit");
    static constexpr unsigned long long value{ 0xd *pow(base,sizeof...(Digits))
                                               + parse_int<base,Digits...>::value};
};
template <unsigned base, char... Digits>
struct parse_int<base,'e',Digits...>{
    static_assert(base>0xe,"invalid digit");
    static constexpr unsigned long long value{ 0xe *pow(base,sizeof...(Digits))
                                               + parse_int<base,Digits...>::value};
};
template <unsigned base, char... Digits>
struct parse_int<base,'f',Digits...>{
    static_assert(base>0xf,"invalid digit");
    static constexpr unsigned long long value{ 0xf *pow(base,sizeof...(Digits))
                                               + parse_int<base,Digits...>::value};
};
template <unsigned base, char... Digits>
struct parse_int<base,'A',Digits...>{
    static_assert(base>0xA,"invalid digit");
    static constexpr unsigned long long value{ 0xa *pow(base,sizeof...(Digits))
                                               + parse_int<base,Digits...>::value};
};
template <unsigned base, char... Digits>
struct parse_int<base,'B',Digits...>{
    static_assert(base>0xB,"invalid digit");
    static constexpr unsigned long long value{ 0xb *pow(base,sizeof...(Digits))
                                               + parse_int<base,Digits...>::value};
};
template <unsigned base, char... Digits>
struct parse_int<base,'C',Digits...>{
    static_assert(base>0xC,"invalid digit");
    static constexpr unsigned long long value{ 0xc *pow(base,sizeof...(Digits))
                                               + parse_int<base,Digits...>::value};
};
template <unsigned base, char... Digits>
struct parse_int<base,'D',Digits...>{
    static_assert(base>0xD,"invalid digit");
    static constexpr unsigned long long value{ 0xd *pow(base,sizeof...(Digits))
                                               + parse_int<base,Digits...>::value};
};
template <unsigned base, char... Digits>
struct parse_int<base,'E',Digits...>{
    static_assert(base>0xE,"invalid digit");
    static constexpr unsigned long long value{ 0xe *pow(base,sizeof...(Digits))
                                               + parse_int<base,Digits...>::value};
};
template <unsigned base, char... Digits>
struct parse_int<base,'F',Digits...>{
    static_assert(base>0xF,"invalid digit");
    static constexpr unsigned long long value{ 0xf *pow(base,sizeof...(Digits))
                                               + parse_int<base,Digits...>::value};
};
}
}
}
#endif /* SUFFIXESPARSENUMBERS_H_ */
\end{codeblock}

\section{integral type fitting}
For its usage, see the implementation of binary literals.
\begin{codeblock}
#ifndef SELECT_INT_TYPE_H_
#define SELECT_INT_TYPE_H_
#include <type_traits>
#include <limits>

namespace std {
namespace suffixes {
namespace select_int_type {

template <unsigned long long val, typename... INTS>
struct select_int_type;

template <unsigned long long val, typename INTTYPE, typename... INTS>
struct select_int_type<val,INTTYPE,INTS...>:conditional<
    val<=static_cast<unsigned long long>(std::numeric_limits<INTTYPE>::max())
    ,INTTYPE
    ,typename select_int_type<val,INTS...>::type >{
    static typename select_int_type::type const
        value{ static_cast<typename select_int_type::type>(val) };
};

template <unsigned long long val>
struct select_int_type<val>{
    typedef unsigned long long type;
    static type const value{  val };
};

}}}
#endif /* SELECT_INT_TYPE_H_ */
\end{codeblock}


\section{binary}
\begin{codeblock}
#ifndef BINARY_H_
#define BINARY_H_
#include <limits>
#include <type_traits>
#include "select_int_type.h"
namespace std{
namespace suffixes{
namespace binary{
namespace __impl{

template <char... Digits>
struct bitsImpl{
    static_assert(! sizeof...(Digits),
            "binary literal digits must be 0 or 1");
    static constexpr unsigned long long value=0;
};

template <char... Digits>
struct bitsImpl<'0',Digits...>{
    static constexpr unsigned long long value=bitsImpl<Digits...>::value;
};

template <char... Digits>
struct bitsImpl<'1',Digits...>{
    static constexpr unsigned long long value=
            bitsImpl<Digits...>::value|(1ULL<<sizeof...(Digits));
};
using std::suffixes::select_int_type::select_int_type;
}


template <char... Digits>
constexpr typename
__impl::select_int_type<__impl::bitsImpl<Digits...>::value,
      int, unsigned, long, unsigned long, long long>::type
operator"" b(){
    return    __impl::select_int_type<__impl::bitsImpl<Digits...>::value,
            int, unsigned, long, unsigned long, long long>::value;
}
template <char... Digits>
constexpr typename
__impl::select_int_type<__impl::bitsImpl<Digits...>::value,
      long, unsigned long, long long>::type
operator"" bl(){
    return    __impl::select_int_type<__impl::bitsImpl<Digits...>::value,
                  long, unsigned long, long long>::value;
}
template <char... Digits>
constexpr auto
operator"" bL() -> decltype(operator "" bl<Digits...>()){
    return     operator "" bl<Digits...>();
}

template <char... Digits>
constexpr typename
__impl::select_int_type<__impl::bitsImpl<Digits...>::value,
       long long>::type
operator"" bll(){
    return     __impl::select_int_type<__impl::bitsImpl<Digits...>::value,
                  long long>::value;
}
template <char... Digits>
constexpr auto
operator"" bLL() -> decltype(operator "" bll<Digits...>()){
    return     operator "" bll<Digits...>();
}

template <char... Digits>
constexpr typename
__impl::select_int_type<__impl::bitsImpl<Digits...>::value,
      unsigned, unsigned long>::type
operator"" bu(){
    return     __impl::select_int_type<__impl::bitsImpl<Digits...>::value,
                  unsigned, unsigned long>::value;
}

template <char... Digits>
constexpr auto
operator"" bU() -> decltype(operator "" bu<Digits...>()){
    return     operator "" bu<Digits...>();
}

template <char... Digits>
constexpr typename
__impl::select_int_type<__impl::bitsImpl<Digits...>::value,
       unsigned long>::type
operator"" bul(){
    return     __impl::select_int_type<__impl::bitsImpl<Digits...>::value,
                  unsigned long>::value;
}
template <char... Digits>
constexpr auto
operator"" bUL() -> decltype(operator "" bul<Digits...>()){
    return     operator "" bul<Digits...>();
}
template <char... Digits>
constexpr auto
operator"" buL() -> decltype(operator "" bul<Digits...>()){
    return     operator "" bul<Digits...>();
}
template <char... Digits>
constexpr auto
operator"" bUl() -> decltype(operator "" bul<Digits...>()){
    return     operator "" bul<Digits...>();
}
template <char... Digits>
constexpr unsigned long long
operator"" bull(){
    return __impl::bitsImpl<Digits...>::value;
}
template <char... Digits>
constexpr unsigned long long
operator"" bULL(){
    return __impl::bitsImpl<Digits...>::value;
}
template <char... Digits>
constexpr unsigned long long
operator"" buLL(){
    return __impl::bitsImpl<Digits...>::value;
}
template <char... Digits>
constexpr unsigned long long
operator"" bUll(){
    return __impl::bitsImpl<Digits...>::value;
}


} // binary
} //suffixes
} // std
#endif /* BINARY_H_ */
\end{codeblock}

%%%%%%%%%%

\section{basic_string}
\begin{codeblock}
#ifndef STRING_SUFFIX_H_
#define STRING_SUFFIX_H_
#include <string>
namespace std{
namespace suffixes{
namespace string{
#if 0 // less typing variant
#define __MAKE_SUFFIX_S(CHAR) \
    basic_string<CHAR>\
operator "" s(CHAR const *str, size_t len){\
    return basic_string<CHAR>(str,len);\
}

__MAKE_SUFFIX_S(char)
__MAKE_SUFFIX_S(wchar_t)
__MAKE_SUFFIX_S(char16_t)
__MAKE_SUFFIX_S(char32_t)
#undef __MAKE_SUFFIX
#else // copy-paste version for proposal

basic_string<char>
operator "" s(char const *str, size_t len){
    return basic_string<char>(str,len);
}
basic_string<wchar_t>
operator "" s(wchar_t const *str, size_t len){
    return basic_string<wchar_t>(str,len);
}
basic_string<char16_t>
operator "" s(char16_t const *str, size_t len){
    return basic_string<char16_t>(str,len);
}
basic_string<char32_t>
operator "" s(char32_t const *str, size_t len){
    return basic_string<char32_t>(str,len);
}

#endif
}
}
}
#endif /* STRING_SUFFIX_H_ */
\end{codeblock}
%%%%%%%%%%%%%%%%%
\section{std::complex}
\begin{codeblock}
namespace std{
namespace suffixes{
namespace complex{
constexpr
std::complex<long double> operator"" _li(long double d){
    return std::complex<long double>{0,d};
}
constexpr
std::complex<long double> operator"" _li(unsigned long long d){
    return std::complex<long double>{0,static_cast<long double>(d)};
}
constexpr
std::complex<long double> operator"" _lr(long double d){
    return std::complex<long double>{d,0};
}
constexpr
std::complex<long double> operator"" _lr(unsigned long long d){
    return std::complex<long double>{static_cast<long double>(d),0};
}
constexpr
std::complex<double> operator"" _i(long double d){
    return std::complex<double>{0,static_cast<double>(d)};
}
constexpr
std::complex<double> operator"" _i(unsigned long long d){
    return std::complex<double>{0,static_cast<double>(d)};
}
constexpr
std::complex<double> operator"" _r(long double d){
    return std::complex<double>{static_cast<double>(d),0};
}
constexpr
std::complex<double> operator"" _r(unsigned long long d){
    return std::complex<double>{static_cast<double>(d),0};
}
constexpr
std::complex<float> operator"" _fi(long double d){
    return std::complex<float>{0,static_cast<float>(d)};
}
constexpr
std::complex<float> operator"" _fi(unsigned long long d){
    return std::complex<float>{0,static_cast<float>(d)};
}
constexpr
std::complex<float> operator"" _fr(long double d){
    return std::complex<float>{static_cast<float>(d),0};
}
constexpr
std::complex<float> operator"" _fr(unsigned long long d){
    return std::complex<float>{static_cast<float>(d),0};
}
}
}
}
\end{codeblock}
\section{duration}

\begin{codeblock}
#ifndef CHRONO_SUFFIX_H_
#define CHRONO_SUFFIX_H_
#include <chrono>
#include <limits>
#include "suffixes_parse_integers.h"
namespace std {
namespace suffixes {
namespace chrono {

namespace __impl {
using namespace std::suffixes::parse_int;

template <unsigned long long val, typename DUR>
struct select_type:
    conditional<val <=
      static_cast<unsigned long long>(std::numeric_limits<typename DUR::rep>::max())
    , DUR
    , void > {
        static constexpr typename select_type::type
            value{ static_cast<typename select_type::type>(val) };
    };
}

template <char... Digits>
constexpr typename 
__impl::select_type<__impl::base_dispatch<Digits...>::value,std::chrono::hours>::type
operator"" h(){
    return  __impl::select_type<__impl::base_dispatch<Digits...>::value,
                std::chrono::hours>::value;
}
constexpr std::chrono::duration<long double, ratio<3600,1>> operator"" h(long double hours){
    return std::chrono::duration<long double,ratio<3600,1>>{hours};
}
template <char... Digits>
constexpr typename 
__impl::select_type<__impl::base_dispatch<Digits...>::value,std::chrono::minutes>::type
operator"" min(){
    return __impl::select_type<__impl::base_dispatch<Digits...>::value,
               std::chrono::minutes>::value;
}
constexpr std::chrono::duration<long double, ratio<60,1>> operator"" min(long double min){
    return std::chrono::duration<long double,ratio<60,1>>{min};
}

template <char... Digits>
constexpr typename 
__impl::select_type<__impl::base_dispatch<Digits...>::value,std::chrono::seconds>::type
operator"" s(){
    return __impl::select_type<__impl::base_dispatch<Digits...>::value,
               std::chrono::seconds>::value;
}
constexpr std::chrono::duration<long double, ratio<1,1>> operator"" s(long double sec){
    return std::chrono::duration<long double,ratio<1,1>>{sec};
}

template <char... Digits>
constexpr typename 
__impl::select_type<__impl::base_dispatch<Digits...>::value,std::chrono::milliseconds>::type
operator"" ms(){
    return __impl::select_type<__impl::base_dispatch<Digits...>::value,
               std::chrono::milliseconds>::value;
}
constexpr std::chrono::duration<long double, ratio<1,1000>> operator"" ms(long double msec){
    return std::chrono::duration<long double,ratio<1,1000>>{msec};
}

template <char... Digits>
constexpr typename 
__impl::select_type<__impl::base_dispatch<Digits...>::value,std::chrono::microseconds>::type
operator"" us(){
    return __impl::select_type<__impl::base_dispatch<Digits...>::value,
               std::chrono::microseconds>::value;
}
constexpr std::chrono::duration<long double, ratio<1,1000000>> operator"" us(long double usec){
    return std::chrono::duration<long double,ratio<1,1000000>>{usec};
}

template <char... Digits>
constexpr typename 
__impl::select_type<__impl::base_dispatch<Digits...>::value,std::chrono::nanoseconds>::type
operator"" ns(){
    return __impl::select_type<__impl::base_dispatch<Digits...>::value,
               std::chrono::nanoseconds>::value;
}
constexpr std::chrono::duration<long double, ratio<1,1000000000>> operator"" ns(long double nsec){
    return std::chrono::duration<long double,ratio<1,1000000000>>{nsec};
}
}
}
}
#endif /* CHRONO_SUFFIX_H_ */
\end{codeblock}



\chapter{Proposed Library Additions}
It must be decided in which section to actually put the proposed changes. I suggest we add them to the corresponding library parts, where appropriate.
\section{namespace suffixes for collecting standard UDLs}
As a common schema this paper proposes to put all suffixes for user defined literals in separate namespaces that are below the namespace \tcode{std::suffixes}. 

\section{Suffixes Integer Parsing Utilities}
Append a subclause [suffixes.parseint] to clause [utilities] and expand the table in [utilities.general] accordingly.
Insert the subclause [suffixes.parseint]

\rSec1[suffixes.parseint]{Parsing integer literals}
\pnum
This subclause contains helper template classes for implementing template user-defined literal operators for parsing  integer literals.

\synopsis{Header \tcode{<suffix_parse>} synopsis}
\begin{codeblock}
namespace std {
namespace suffixes {

template <char... Digits>
struct base_dispatch;

template <unsigned base, char... Digits>
struct parse_int;

} // suffixes

constexpr unsigned long long
pow(unsigned base, size_t to);

} // std
\end{codeblock}

\pnum
The class templates \tcode{base_dispatch} and \tcode{parse_int} are intended to be used by implementors of a user-defined-literal operator template for parsing integral values. They both provide a \tcode{static constexpr unsigned long long} member variable named \tcode{value}. 
\pnum
Class template \tcode{parse_int} can be used when the numerical base is known and not to be determined from \tcode{Digits...}.
\pnum
The class template \tcode{base_dispatch} will determine the numerical base to be used like the compiler does with integral literals, i.e., a number starting with \tcode{0} (zero) will be considered octal, a number starting with \tcode{0x} or \tcode{0X} will be considered hexadecimal and all other numbers will be considered decimal.
\pnum
The compile-time evaluated function \tcode{pow} is used to perform the positional calculation $base^{power}$ within \tcode{parse_int}.

\enterexample
\begin{codeblock}
template <char... Digits>
constexpr  unsigned long long
operator"" _testit(){
    return std::suffixes::base_dispatch<Digits...>::value;
}

constexpr auto a = 123_testit;   // value 123
constexpr auto b = 0123_testit;  // value 0123
constexpr auto c = 0x123_testit; // value 0x123

\end{codeblock}
\exitexample

\begin{itemdecl}
template <char... Digits>
struct base_dispatch;
\end{itemdecl}

\begin{itemdescr}
\pnum
\effects
Creates an integral constant as its \tcode{static constexpr} member variable \tcode{value} which is of type \tcode{unsigned long long} with the value determined from the first or first two \tcode{char} values in \tcode{Digits...} as described below.

\pnum
If \tcode{Digits...} starts with \tcode{'0', 'x'} or \tcode{'0', 'X'} the remaining characters are parsed as a hexadecimal number. If there are no such subsequent characters or \tcode{Digits...} contains characters beyond those for hexadecimal literals the program is ill formed.

\pnum
If \tcode{Digits...} starts with \tcode{'0'} the remaining characters are parsed as an octal number. If there are characters in \tcode{Digits...} that are not octal digits ('0'-'7') the program is ill-formed.

\pnum
Otherwise \tcode{Digits...} is interpreted as a decimal number. If there are characters in \tcode{Digits...} that are not decimal digits, the program is ill formed.
\end{itemdescr}

\begin{itemdecl}
template <unsigned base, char... Digits>
struct parse_int;
\end{itemdecl}

\begin{itemdescr}
\pnum
\precondition
\tcode{base > 1 \&\& base <= 16}
\pnum
\effects
Creates an integral constant as its \tcode{static constexpr} member variable \tcode{value} which is of type \tcode{unsigned long long} with the value determined by parsing \tcode{Digits...} as an integer literal of the given \tcode{base}.

\pnum
If \tcode{Digits...} contains characters outside the digits of the given base, the program is ill-formed.

\enterexample
\begin{codeblock}
template <char... Digits>
constexpr  unsigned long long
operator"" _ternary(){
    return std::suffixes::parse_int<3,Digits...>::value;
}
constexpr auto three= 010_ternary;
static_assert(three==3, "_ternary should be three-based");
constexpr auto invalid=3_ternary; // ill-formed.
\end{codeblock}
\exitexample
\end{itemdescr}

\begin{itemdecl}
constexpr unsigned long long
pow(unsigned base, size_t to) ;
\end{itemdecl}

\begin{itemdescr}
\pnum
\precondition
\tcode{base > 1 \&\& base <=16}

\pnum
\returns \tcode{to ? base * pow(base,to-1) : 1}

\pnum
Overflow is undetected.
\end{itemdescr}
%%%%
\section{ Integer Constant Type Selection}
Append a subclause [suffixes.selectinttype] to clause [utilities] and expand the table in [utilities.general] accordingly.
Insert the subclause [suffixes.selectinttype]

\rSec1[suffixes.selectinttype]{Select Matching Type for Integer Constants}
\pnum
This subclause contains helper template classes for performing the determination of the type of integer literals as in clause [lex.icon] p 2.

\synopsis{Header \tcode{<select_int_type>} synopsis}
\begin{codeblock}
namespace std {

template <unsigned long long val, typename... INTS>
struct select_int_type;
}
\end{codeblock}

\enterexample
The class template will be used with a list of integral types as those given in clause [lex.icon] p 2 in one of the table 6. The following code demonstrates its use.
\begin{codeblock}
using std::select_int_type;
template <unsigned long long val>
constexpr
typename select_int_type<val,
short, int, long long>::type
foo() {
    return  select_int_type<val,
            short, int, long long>::value;
}
static_assert(std::is_same<decltype(foo<100>()), short>::value,"foo<100>() is short");
static_assert(std::is_same<decltype(foo<0x10000>()), int>::value,"foo<0x10000>() is int");
static_assert(std::is_same<decltype(foo<0x100000000000>()), long long>::value,"foo<0x100000000000>() is long long");
\end{codeblock}

\exitexample
\begin{itemdecl}
template <unsigned long long val, typename... INTS>
struct select_int_type;
\end{itemdecl}

\begin{itemdescr}
\pnum
\requires
\tcode{INTS...} must consist of a list of integral types \tcode{T}$_i$ where \tcode{numeric_limits<T}$_i$\tcode{>::max()} is less or equal of \tcode{numeric_limits<T}$_{i+1}$\tcode{>::max()}.

\pnum
\effects
The member \tcode{type} of \tcode{select_int_type} corresponds to the first type \tcode{T} in \tcode{INTS...} where \tcode{val <= static_cast<unsigned long long>(numeric_limits<T>::max())}. If no such type exists the member \tcode{type} is \tcode{unsigned long long}.

\pnum
The member \tcode{value} is of type \tcode{select_int_type::type} and its value is {val}.

\enternote
No overflow can occur.
\exitnote 

\end{itemdescr}
%%%%%
\section{operator"" b() etc. for binary integer literals}
Append a subclause [suffixes.binary] to clause [utilities] and expand the table in [utilities.general] accordingly.

Insert the subclause [suffixes.binary]

\rSec1[suffixes.binary]{Binary integer literals}

\pnum
This subclause contains user-defined literal operators for representing binary encoded integer literals.

\synopsis{Header \tcode{<suffix_binary>} synopsis}
\begin{codeblock}
namespace std{
namespace suffixes{
namespace binary{

template <char... Digits>
constexpr @\seebelow@ 
operator"" b();
template <char... Digits>
constexpr @\seebelow@ 
operator"" bu();
template <char... Digits>
constexpr @\seebelow@ 
operator"" bU();
template <char... Digits>
constexpr @\seebelow@ 
operator"" bl();
template <char... Digits>
constexpr @\seebelow@ 
operator"" bL();
template <char... Digits>
constexpr @\seebelow@ 
operator"" bul();
template <char... Digits>
constexpr @\seebelow@ 
operator"" buL();
template <char... Digits>
constexpr @\seebelow@ 
operator"" bll();
template <char... Digits>
constexpr @\seebelow@ 
operator"" bLL();
template <char... Digits>
constexpr unsigned long long
operator"" bull();
template <char... Digits>
constexpr unsigned long long
operator"" buLL();
template <char... Digits>
constexpr unsigned long long
operator"" bUll();
template <char... Digits>
constexpr unsigned long long
operator"" bULL();
} // binary
} //suffixes
} // std
\end{codeblock}

\pnum
A binary integer literal is a sequence of the binary digits \tcode{'0'} (zero) or \tcode{'1'} (one) that is followed by one of the suffixes in namespace \tcode{std::suffixes::binary}. If there is any other digit in a binary integer literal the program is ill-formed. The lexically first digit of the sequence of digits is the most significant. The sequence of binary digits forming a binary literal create an integral value that corresponds to its interpretation as a binary number.

\pnum
The type of the binary literal is determined from the its value and the additional suffix (\tcode{u, l, ul, ull} and their uppercase variants as with other integer literals) to \tcode{b} like the determination of the type of octal integer literals in clause [lex.icon] p 2.

\pnum
\enterexample 
The following code shows some binary literals. The type of xll is adjusted due to its large value, assuming \tcode{sizeof(long) < 8} and \tcode{char} as octet.
\begin{codeblock}
{
    using namespace std::suffixes::binary;
    auto x=100b; // int x=4
    auto y=1000bu; // unsigned y=8
    auto xll=0111111111111111111111111111111111111111111111111111111111111111b;
         // long long xll
}
\end{codeblock}
\exitexample

\begin{itemdecl}
template <char... Digits>
constexpr @\seebelow@ 
operator"" b();
\end{itemdecl}

\begin{itemdescr}
\pnum
\effects
Creates an integral constant with the value determined as described above. The return type is determined according to the first row of table 6 in clause [lex.icon] p 2 and the column for octal literals. 
\end{itemdescr}

\begin{itemdecl}
template <char... Digits>
constexpr @\seebelow@ 
operator"" bu();
template <char... Digits>
constexpr @\seebelow@ 
operator"" bU();
\end{itemdecl}

\begin{itemdescr}
\pnum
\effects
Creates an integral value determined as described above. The return type is determined according to the second row of table 6 in clause [lex.icon] p 2 and the column for octal literals. 
\end{itemdescr}

\begin{itemdecl}
template <char... Digits>
constexpr @\seebelow@ 
operator"" bl();
template <char... Digits>
constexpr @\seebelow@ 
operator"" bL();
\end{itemdecl}

\begin{itemdescr}
\pnum
\effects
Creates an integral value determined as described above. The return type is determined according to the third row of table 6 in clause [lex.icon] p 2 and the column for octal literals. 
\end{itemdescr}

\begin{itemdecl}
template <char... Digits>
constexpr @\seebelow@ 
operator"" bul();
template <char... Digits>
constexpr @\seebelow@ 
operator"" buL();
template <char... Digits>
constexpr @\seebelow@ 
operator"" bUl();
template <char... Digits>
constexpr @\seebelow@ 
operator"" bUL();
\end{itemdecl}

\begin{itemdescr}
\pnum
\effects
Creates an integral value determined as described above. The return type is determined according to the fourth row of table 6 in clause [lex.icon] p 2 and the column for octal literals. 
\end{itemdescr}

\begin{itemdecl}
template <char... Digits>
constexpr @\seebelow@ 
operator"" bll();
template <char... Digits>
constexpr @\seebelow@ 
operator"" bLL();
\end{itemdecl}

\begin{itemdescr}
\pnum
\effects
Creates an integral constant with the value determined as described above. The return type is determined according to the fifth row of table 6 in clause [lex.icon] p 2 and the column for octal literals. 
\end{itemdescr}
\begin{itemdecl}
template <char... Digits>
constexpr unsigned long long
operator"" bull();
template <char... Digits>
constexpr unsigned long long
operator"" buLL();
template <char... Digits>
constexpr unsigned long long
operator"" bUll();
template <char... Digits>
constexpr unsigned long long
operator"" bULL();
\end{itemdecl}

\begin{itemdescr}
\pnum
\effects
Creates an integral value determined as described above. 
\end{itemdescr}

\section{operator"" s() for basic_string}
Make the following additions and changes to library clause 21 [strings] to accommodate the user-defined literal suffix s for string literals resulting in a corresponding string object instead of array of characters.

Insert in 21.3 [string.classes] in the synopsis at the appropriate place the namespace std::suffixes::string
\begin{codeblock}
namespace std{
namespace suffixes{
namespace string{
basic_string<char> operator "" s(char const *str, size_t len);
basic_string<wchar_t> operator "" s(wchar_t const *str, size_t len);
basic_string<char16_t> operator "" s(char16_t const *str, size_t len);
basic_string<char32_t> operator "" s(char32_t const *str, size_t len);
}}}
\end{codeblock}

Before subclause 21.7 [c.strings] add a new subclause [basic.string.suffixes]

\rSec1[basic.string.suffixes]{Suffix for basic_string literals}
\begin{itemdecl}
basic_string<char> operator "" s(char const *str, size_t len);
\end{itemdecl}

\begin{itemdescr}
\pnum
\returns
\tcode{basic_string<char>\{str,len\}}
\end{itemdescr}

\begin{itemdecl}
basic_string<wchar_t> operator "" s(wchar_t const *str, size_t len);
\end{itemdecl}
\begin{itemdescr}
\pnum
\returns
\tcode{basic_string<wchar_t>\{str,len\}}
\end{itemdescr}

\begin{itemdecl}
basic_string<char16_t> operator "" s(char16_t const *str, size_t len);
\end{itemdecl}
\begin{itemdescr}
\pnum
\returns
\tcode{basic_string<char16_t>\{str,len\}}
\end{itemdescr}

\begin{itemdecl}
basic_string<char32_t> operator "" s(char32_t const *str, size_t len);
\end{itemdecl}
\begin{itemdescr}
\pnum
\returns
\tcode{basic_string<char32_t>\{str,len\}}
\end{itemdescr}

\section{UDL operators for std::complex}
Make the following additions and changes to library subclause 26.4 [complex.numbers] to accommodate user-defined literal suffixes for complex number literals.

Insert in subclause 26.4.1 [complex.syn] in the synopsis at the appropriate place the namespace std::suffixes::complex
\begin{codeblock}
namespace std{
namespace suffixes{
namespace complex{
constexpr std::complex<long double> operator"" li(long double);
constexpr std::complex<long double> operator"" LI(long double);
constexpr std::complex<long double> operator"" li(unsigned long long);
constexpr std::complex<long double> operator"" LI(unsigned long long);
constexpr std::complex<long double> operator"" lr(long double);
constexpr std::complex<long double> operator"" LR(long double);
constexpr std::complex<long double> operator"" lr(unsigned long long);
constexpr std::complex<long double> operator"" LR(unsigned long long);
constexpr std::complex<double> operator"" i(long double);
constexpr std::complex<double> operator"" I(long double);
constexpr std::complex<double> operator"" i(unsigned long long);
constexpr std::complex<double> operator"" I(unsigned long long);
constexpr std::complex<double> operator"" r(long double);
constexpr std::complex<double> operator"" R(long double);
constexpr std::complex<double> operator"" r(unsigned long long);
constexpr std::complex<double> operator"" R(unsigned long long);
constexpr std::complex<float> operator"" fi(long double);
constexpr std::complex<float> operator"" FI(long double);
constexpr std::complex<float> operator"" fi(unsigned long long);
constexpr std::complex<float> operator"" FI(unsigned long long);
constexpr std::complex<float> operator"" fr(long double);
constexpr std::complex<float> operator"" FR(long double);
constexpr std::complex<float> operator"" fr(unsigned long long);}
constexpr std::complex<float> operator"" FR(unsigned long long);}
}
}
\end{codeblock}

Append a new subclause after subclause 26.4.10 [ccmplx] as follows
\rSec1[complex.suffixes]{Suffix for complex number literals}
\pnum
This section describes literal suffixes for constructing complex number literals. The suffixes i, li, fi create complex numbers with their imaginary part denoted by the given literal number and the real part being zero of the types \tcode{complex<double>}, \tcode{complex<long double>}, and \tcode{complex<float>} respectively. 

The suffixes r, lr, fr create complex numbers with the real part denoted by the given literal number and the imaginary part being zero of the types \tcode{complex<double>}, \tcode{complex<long double>}, and \tcode{complex<float>} respectively. 

\begin{itemdecl}
constexpr std::complex<long double> operator"" li(long double d);
constexpr std::complex<long double> operator"" LI(long double d);
constexpr std::complex<long double> operator"" li(unsigned long long d);
constexpr std::complex<long double> operator"" LI(unsigned long long d);
\end{itemdecl}

\begin{itemdescr}
\pnum
\effects
Creates a complex literal as \tcode{std::complex<long double>\{0.0L, static_cast<long double>(d)\}}.
\end{itemdescr}

\begin{itemdecl}
constexpr std::complex<double> operator"" i(long double d);
constexpr std::complex<double> operator"" I(long double d);
constexpr std::complex<double> operator"" i(unsigned long long d);
constexpr std::complex<double> operator"" I(unsigned long long d);
\end{itemdecl}

\begin{itemdescr}
\pnum
\effects
Creates a complex literal as \tcode{std::complex<double>\{0.0, static_cast<double>(d)\}}.
\end{itemdescr}

\begin{itemdecl}
constexpr std::complex<float> operator"" fi(long double d);
constexpr std::complex<float> operator"" FI(long double d);
constexpr std::complex<float> operator"" fi(unsigned long long d);
constexpr std::complex<float> operator"" FI(unsigned long long d);
\end{itemdecl}

\begin{itemdescr}
\pnum
\effects
Creates a complex literal as \tcode{std::complex<float>\{0.0f, static_cast<float>(d)\}}.
\end{itemdescr}

\begin{itemdecl}
constexpr std::complex<long double> operator"" lr(long double d);
constexpr std::complex<long double> operator"" LR(long double d);
constexpr std::complex<long double> operator"" lr(unsigned long long d);
constexpr std::complex<long double> operator"" LR(unsigned long long d);
\end{itemdecl}

\begin{itemdescr}
\pnum
\effects
Creates a complex literal as \tcode{std::complex<long double>\{static_cast<long double>(d), 0.0L\}}.
\end{itemdescr}
\begin{itemdecl}
constexpr std::complex<double> operator"" r(long double d);
constexpr std::complex<double> operator"" R(long double d);
constexpr std::complex<double> operator"" r(unsigned long long d);
constexpr std::complex<double> operator"" R(unsigned long long d);
\end{itemdecl}

\begin{itemdescr}
\pnum
\effects
Creates a complex literal as \tcode{std::complex<double>\{static_cast<double>(d), 0.0\}}.
\end{itemdescr}

\begin{itemdecl}
constexpr std::complex<float> operator"" fr(long double d);
constexpr std::complex<float> operator"" FR(long double d);
constexpr std::complex<float> operator"" fr(unsigned long long d);}
constexpr std::complex<float> operator"" FR(unsigned long long d);}
\end{itemdecl}

\begin{itemdescr}
\pnum
\effects
Creates a complex literal as \tcode{std::complex<float>\{static_cast<float>(d), 0.0f\}}.
\end{itemdescr}




\section{Suffixes for std::chrono::duration values}
Make the following additions and changes to library subclause 20.11 [time] to accommodate user-defined literal suffixes for chrono::duration literals.

Insert in subclause 20.11.2 [time.syn] in the synopsis at the appropriate place the namespace std::suffixes::chrono
\begin{codeblock}
namespace std {
namespace suffixes {
namespace chrono {
constexpr 
std::chrono::hours operator"" h(unsigned long long);
constexpr 
std::chrono::duration<@\seebelow@, ratio<3600,1>> operator"" h(long double);
constexpr 
std::chrono::minutes operator"" min(unsigned long long);
constexpr 
std::chrono::duration<@\seebelow@, ratio<60,1>> operator"" min(long double);
constexpr 
std::chrono::seconds operator"" s(unsigned long long);
constexpr 
std::chrono::duration<@\seebelow@, ratio<1,1>> operator"" s(long double);
constexpr 
std::chrono::milliseconds operator"" ms(unsigned long long);
constexpr 
std::chrono::duration<@\seebelow@, ratio<1,1000>> operator"" ms(long double);
constexpr 
std::chrono::microseconds operator"" us(unsigned long long);
constexpr 
std::chrono::duration<@\seebelow@, ratio<1,1000000>> operator"" us(long double);
constexpr 
std::chrono::nanoseconds operator"" ns(unsigned long long);
constexpr 
std::chrono::duration<@\seebelow@, ratio<1,1000000000>> operator"" ns(long double);
}
}
}
\end{codeblock}

Insert in subclause 20.11.5 [time.duration] after subclause 20.11.5.7 [time.duration.cast] a new subclause 20.11.5.8 [time.duration.suffixes] as follows.

\rSec2[time.duration.suffixes]{Suffix for duration literals}
\pnum
This section describes literal suffixes for constructing duration literals. The suffixes \tcode{h,min,s,ms,us,ns} denote duration values of the corresponding types \tcode{hours}, \tcode{minutes}, \tcode{seconds}, \tcode{miliseconds}, \tcode{microseconds}, and \tcode{nanoseconds} respectively if they are applied to integral literals. 

\pnum
If the above suffixes are applied to a floating point literal the result is a std::duration literal with an implementation-defined floating point representation.

\pnum
If the above suffixes are applied to an integer literal and the resulting \tcode{chrono::duration} value cannot be represented in the result type because of overflow, the program is ill-formed.

\pnum
\enterexample 
The following code shows some duration literals.
\begin{codeblock}
{
    using namespace std::suffixes::chrono;
    auto constexpr aday=24h; 
    auto constexpr lesson=45min; 
    auto constexpr halfanhour=0.5h;
}
\end{codeblock}
\exitexample

\pnum
\enternote
The suffix for microseconds is \tcode{us}, but if unicode identifiers are allowed implementations are encouraged to provide $\mu{}$\tcode{s} as well.
\exitnote

\begin{itemdecl}
constexpr 
std::chrono::hours operator"" h(unsigned long long hours);
constexpr 
std::chrono::duration<@\seebelow@, ratio<3600,1>> operator"" h(long double hours);
\end{itemdecl}

\begin{itemdescr}
\pnum
\effects
Creates a \tcode{duration} literal representing \tcode{hours} hours.
\end{itemdescr}

\begin{itemdecl}
constexpr 
std::chrono::minutes operator"" min(unsigned long long min);
constexpr 
std::chrono::duration<@\seebelow@, ratio<60,1>> operator"" min(long double min);
\end{itemdecl}

\begin{itemdescr}
\pnum
\effects
Creates a \tcode{duration} literal representing \tcode{min} minutes.
\end{itemdescr}

\begin{itemdecl}
constexpr 
std::chrono::seconds operator"" s(unsigned long long sec);
constexpr 
std::chrono::duration<@\seebelow@, ratio<1,1>> operator"" s(long double sec);
\end{itemdecl}

\begin{itemdescr}
\pnum
\effects
Creates a \tcode{duration} literal representing \tcode{sec} seconds.

\enternote
The same suffix \tcode{s} is used for \tcode{std::basic_string} but there is no conflict, since duration suffixes always apply to numbers and string literal suffixes always apply to character array literals.
\exitnote
\end{itemdescr}

\begin{itemdecl}
constexpr 
std::chrono::milliseconds operator"" ms(unsigned long long msec);
constexpr 
std::chrono::duration<@\seebelow@, ratio<1,1000>> operator"" ms(long double msec);
\end{itemdecl}

\begin{itemdescr}
\pnum
\effects
Creates a \tcode{duration} literal representing \tcode{msec} milliseconds.
\end{itemdescr}

\begin{itemdecl}
constexpr 
std::chrono::microseconds operator"" us(unsigned long long usec);
constexpr 
std::chrono::duration<@\seebelow@, ratio<1,1000000>> operator"" us(long double usec);
\end{itemdecl}

\begin{itemdescr}
\pnum
\effects
Creates a \tcode{duration} literal representing \tcode{usec} microseconds.
\end{itemdescr}

\begin{itemdecl}
constexpr 
std::chrono::nanoseconds operator"" ns(unsigned long long nsec);
constexpr 
std::chrono::duration<@\seebelow@, ratio<1,1000000000>> operator"" ns(long double nsec);
\end{itemdecl}

\begin{itemdescr}
\pnum
\effects
Creates a \tcode{duration} literal representing \tcode{nsec} nanoseconds.
\end{itemdescr}


\end{document}   